\documentclass[11pt, a4paper]{article}

% --- UNIVERSAL PREAMBLE BLOCK ---
\usepackage[a4paper, top=2.5cm, bottom=2.5cm, left=2cm, right=2cm]{geometry}
\usepackage{fontspec}
\usepackage{xcolor}
\usepackage{hyperref}
\usepackage{listings}
\usepackage{titlesec}
\usepackage{longtable}
\usepackage{booktabs}

\usepackage[english, provide=*]{babel}
\babelprovide[import, onchar=ids fonts]{english}

% Set default font to Noto Sans as per protocol
\babelfont{rm}{Noto Sans}
\babelfont{tt}{Noto Sans Mono}

% Hyperlink Setup
\hypersetup{
    colorlinks=true,
    linkcolor=blue!60!black,
    filecolor=magenta,
    urlcolor=cyan,
    pdftitle={Windows Server Deployment Guide on AWS},
    pdfauthor={AWS Documentation}
}

% Code Listing Configuration
\definecolor{codegreen}{rgb}{0,0.6,0}
\definecolor{codegray}{rgb}{0.5,0.5,0.5}
\definecolor{codepurple}{rgb}{0.58,0,0.82}
\definecolor{backcolour}{rgb}{0.96,0.96,0.96}

\lstset{
    backgroundcolor=\color{backcolour},
    commentstyle=\color{codegreen},
    keywordstyle=\color{blue},
    numberstyle=\tiny\color{codegray},
    stringstyle=\color{codepurple},
    basicstyle=\ttfamily\footnotesize,
    breakatwhitespace=false,
    breaklines=true,
    captionpos=b,
    keepspaces=true,
    numbers=left,
    numbersep=5pt,
    showspaces=false,
    showstringspaces=false,
    showtabs=false,
    tabsize=2,
    frame=single,
    rulecolor=\color{black!10},
    columns=fullflexible
}

% Define generic style for PowerShell/Bash
\lstdefinelanguage{PowerShell}{
  keywords={Install-WindowsFeature, New-SmbShare, Enable-ComputerRestore, vssadmin, netsh, Add-DnsServerPrimaryZone, Add-DnsServerForwarder, Add-DhcpServerInDC, Add-DhcpServerv4Scope, Set-DhcpServerv4OptionValue, Install-RemoteAccess, New-RDSessionCollection, New-Website, New-WebAppPool, Setup.exe, New-WBPolicy, New-Cluster, New-WebFtpSite, docker, aws, Rename-Computer, New-NetIPAddress, Import-Module, Install-ADDSForest, Install-ADDSDomainController, Get-Service, New-ADOrganizationalUnit, Set-VpnServerConfiguration, New-RDRemoteApp, New-WebBinding, Set-ItemProperty, New-EC2Snapshot, Test-Cluster, Set-ClusterQuorum, Add-ClusterServerRole, Set-WebConfigurationProperty, New-SelfSignedCertificate, New-LocalUser, Get-Acl, Set-Acl, New-Object, Install-Module, Install-Package, Restart-Computer, Get-WindowsFeature, Get-NetAdapter, Set-DnsClientServerAddress, dfsrdiag, repadmin, w32tm, Restart-Service, Set-ADDefaultDomainPasswordPolicy, Enable-ADOptionalFeature},
  keywordstyle=\color{blue}\bfseries,
  ndkeywords={-Name, -IncludeManagementTools, -Path, -FullAccess, -ReadAccess, -Drive, -IPAddress, -ScopeId, -Router, -DnsServer, -VpnType, -CollectionName, -SessionHost, -Port, -PhysicalPath, -ApplicationPool, -Protocol, -SslFlags, -VolumePath, -Node, -StaticAddress, -NoStorage, -DomainName, -DatabasePath, -Identity, -Scope, -Target, -Confirm, -ComplexityEnabled, -LockoutDuration, -LockoutThreshold, -MaxPasswordAge, -MinPasswordAge, -MinPasswordLength, -PasswordHistoryCount},
  ndkeywordstyle=\color{teal}\bfseries,
  identifierstyle=\color{black},
  sensitive=false,
  comment=[l]{\#},
  morestring=[b]",
  morestring=[b]'
}

\title{\textbf{Windows Server Deployment Guide on AWS Cloud}}
% \author{System Administrator Guide}
\author{Sou Chanrojame, Orn Pheakdey}
% \date{\today}
 
\begin{document}

\maketitle

\begin{abstract}
	This document provides a comprehensive step-by-step guide for deploying various Windows Server roles and services on the AWS Cloud infrastructure, including configuration details for EC2 instances, security groups, and storage optimization.
\end{abstract}

\tableofcontents
\newpage

\section{Prerequisites}

\subsection{AWS Account Setup}
\begin{itemize}
	\item Active AWS account with appropriate permissions
	\item VPC configured with public and private subnets
	\item Security groups properly configured
	\item Key pairs created for RDP access
	\item IAM roles for EC2 instances
\end{itemize}

\subsection{General Windows Server Launch Steps}
\begin{enumerate}
	\item Navigate to EC2 Dashboard in AWS Console
	\item Click ``Launch Instance''
	\item Select Windows Server AMI (2019/2022 recommended)
	\item Choose instance type based on workload
	\item Configure instance details (VPC, subnet, IAM role)
	\item Add storage as needed
	\item Configure security groups
	\item Review and launch with key pair
\end{enumerate}

\newpage

\section{File Server}

\subsection{AWS Configuration}
\begin{description}
	\item[Instance Type:] t3.medium or larger
	\item[Storage:] EBS volumes with provisioned IOPS for performance
	\item[Security Group Ports:] 445 (SMB), 139 (NetBIOS), 3389 (RDP)
\end{description}

\subsection{Implementation Steps}

\subsubsection*{1. Launch Windows Server EC2 Instance}
Select Windows Server 2022 Datacenter and attach additional EBS volumes for file storage.

\subsubsection*{2. Install File Server Role}
\begin{lstlisting}[language=PowerShell]
Install-WindowsFeature -Name FS-FileServer -IncludeManagementTools
Install-WindowsFeature -Name FS-DFS-Namespace, FS-DFS-Replication
\end{lstlisting}

\subsubsection*{3. Configure Storage}
Initialize and format additional EBS volumes and create shared folders.
\begin{lstlisting}[language=PowerShell]
New-SmbShare -Name "SharedFiles" -Path "D:\Shares" -FullAccess "Domain\Admins" -ReadAccess "Domain\Users"
\end{lstlisting}

\subsubsection*{4. Enable Shadow Copies}
\begin{lstlisting}[language=PowerShell]
Enable-ComputerRestore -Drive "D:\"
vssadmin resize shadowstorage /for=D: /on=D: /maxsize=20%
\end{lstlisting}

\subsubsection*{5. Configure AWS Backup}
Create a backup plan for EBS volumes and set retention policies.

\subsection{Best Practices}
\begin{itemize}
	\item Use AWS Storage Gateway for hybrid scenarios
	\item Implement Amazon FSx for Windows File Server for a managed solution
	\item Enable encryption at rest using AWS KMS
	\item Configure NTFS permissions and share permissions
\end{itemize}

\section{Proxy Server (Caching, Control Access)}

\subsection{AWS Configuration}
\begin{description}
	\item[Instance Type:] t3.medium
	\item[Security Group Ports:] 8080, 3128 (proxy), 3389 (RDP)
\end{description}

\subsection{Implementation Steps}
\begin{enumerate}
	\item \textbf{Launch Windows Server Instance}
	\item \textbf{Install Proxy Server Software}
	      \begin{itemize}
		      \item \textbf{Option A (WinGate):} Download/install WinGate and configure proxy settings.
		      \item \textbf{Option B (Squid):} Download Squid for Windows and configure \texttt{squid.conf}.
	      \end{itemize}
	\item \textbf{Configure Proxy Settings}
	      \begin{lstlisting}[language=PowerShell]
# Example configuration for basic proxy
netsh winhttp set proxy proxy-server="localhost:8080" bypass-list="*.local"
\end{lstlisting}
	\item \textbf{Set Up Caching:} Configure cache directory on separate EBS volume and set policies.
	\item \textbf{Access Control:} Configure authentication (AD integration), URL filtering, and blacklists.
	\item \textbf{Configure AWS Security Group:} Allow inbound traffic from specific CIDR blocks only.
\end{enumerate}

\section{DNS Server}

\subsection{AWS Configuration}
\begin{description}
	\item[Instance Type:] t3.small
	\item[Security Group Ports:] 53 (TCP/UDP), 3389 (RDP)
\end{description}

\subsection{Implementation Steps}
\begin{enumerate}
	\item Launch Windows Server Instance in a private subnet.
	\item \textbf{Install DNS Server Role:}
	      \begin{lstlisting}[language=PowerShell]
Install-WindowsFeature -Name DNS -IncludeManagementTools
\end{lstlisting}
	\item \textbf{Configure DNS Zones:}
	      \begin{lstlisting}[language=PowerShell]
# Create Primary Zone
Add-DnsServerPrimaryZone -Name "yourdomain.local" -ReplicationScope "Forest" -PassThru

# Create Reverse Lookup Zone
Add-DnsServerPrimaryZone -NetworkID "10.0.0.0/16" -ReplicationScope "Forest"
\end{lstlisting}
	\item \textbf{Configure Forwarders:}
	      \begin{lstlisting}[language=PowerShell]
# Use AWS DNS or external DNS
Add-DnsServerForwarder -IPAddress "8.8.8.8", "8.8.4.4"
\end{lstlisting}
	\item Integrate with AWS Route 53 Resolver endpoints if needed.
\end{enumerate}

\section{DHCP Server}
\textit{Note: AWS VPC provides DHCP by default; a custom DHCP server is optional.}

\subsection{AWS Configuration}
\begin{description}
	\item[Instance Type:] t3.small
\end{description}

\subsection{Implementation Steps}
\begin{enumerate}
	\item \textbf{Install DHCP Server Role:}
	      \begin{lstlisting}[language=PowerShell]
Install-WindowsFeature -Name DHCP -IncludeManagementTools
Add-DhcpServerInDC -DnsName "dhcp.yourdomain.local"
\end{lstlisting}
	\item \textbf{Configure DHCP Scope:}
	      \begin{lstlisting}[language=PowerShell]
Add-DhcpServerv4Scope -Name "Internal Network" -StartRange 10.0.1.100 -EndRange 10.0.1.200 -SubnetMask 255.255.255.0

Set-DhcpServerv4OptionValue -ScopeId 10.0.1.0 -Router 10.0.1.1
Set-DhcpServerv4OptionValue -ScopeId 10.0.1.0 -DnsServer 10.0.1.10
\end{lstlisting}
	\item \textbf{Configure Reservations:}
	      \begin{lstlisting}[language=PowerShell]
Add-DhcpServerv4Reservation -ScopeId 10.0.1.0 -IPAddress 10.0.1.50 -ClientId "00-11-22-33-44-55" -Description "Print Server"
\end{lstlisting}
	\item \textbf{Authorize DHCP Server:}
	      \begin{lstlisting}[language=PowerShell]
Add-DhcpServerInDC -DnsName "dhcp.yourdomain.local" -IPAddress 10.0.1.10
\end{lstlisting}
\end{enumerate}

\section{VPN Server}

\subsection{AWS Configuration}
\begin{description}
	\item[Instance Type:] t3.small to t3.medium
	\item[Security Group Ports:] 1723 (PPTP), 1701 (L2TP), 500/4500 (IPSec), 443 (SSTP)
	\item[Elastic IP:] Required
\end{description}

\subsection{Implementation Steps}
\begin{enumerate}
	\item Launch Windows Server Instance with Elastic IP.
	\item \textbf{Install Remote Access Role:}
	      \begin{lstlisting}[language=PowerShell]
Install-WindowsFeature -Name RemoteAccess -IncludeManagementTools
Install-WindowsFeature -Name DirectAccess-VPN -IncludeManagementTools
Install-WindowsFeature -Name Routing -IncludeManagementTools
\end{lstlisting}
	\item \textbf{Configure VPN Server:}
	      \begin{lstlisting}[language=PowerShell]
Install-RemoteAccess -VpnType Vpn
\end{lstlisting}
	\item Enable SSTP, L2TP/IPSec, or IKEv2 and configure authentication.
	\item \textbf{Set Up IP Address Assignment:}
	      \begin{lstlisting}[language=PowerShell]
Set-VpnServerConfiguration -TunnelType SSTP -PassThru
\end{lstlisting}
	\item Configure Routing (NAT and tables).
\end{enumerate}

\section{Terminal Server (Thin Clients)}

\subsection{AWS Configuration}
\begin{description}
	\item[Instance Type:] t3.xlarge or larger
	\item[Security Group Ports:] 3389 (RDP), 3391 (RD Gateway)
\end{description}

\subsection{Implementation Steps}
\begin{enumerate}
	\item \textbf{Install RDS Roles:}
	      \begin{lstlisting}[language=PowerShell]
Install-WindowsFeature -Name RDS-RD-Server -IncludeManagementTools
Install-WindowsFeature -Name RDS-Connection-Broker -IncludeManagementTools
Install-WindowsFeature -Name RDS-Web-Access -IncludeManagementTools
Install-WindowsFeature -Name RDS-Gateway -IncludeManagementTools
Install-WindowsFeature -Name RDS-Licensing -IncludeManagementTools
\end{lstlisting}
	\item Configure RDS Deployment via Server Manager.
	\item \textbf{Configure Session Collections:}
	      \begin{lstlisting}[language=PowerShell]
New-RDSessionCollection -CollectionName "Production" -SessionHost "rdsh01.yourdomain.local" -ConnectionBroker "rdcb.yourdomain.local"
\end{lstlisting}
	\item \textbf{Set Up RemoteApp:}
	      \begin{lstlisting}[language=PowerShell]
New-RDRemoteApp -CollectionName "Production" -DisplayName "Microsoft Word" -FilePath "C:\Program Files\Microsoft Office\root\Office16\WINWORD.EXE"
\end{lstlisting}
\end{enumerate}

\section{Web Server}

\subsection{AWS Configuration}
\begin{description}
	\item[Instance Type:] t3.medium
	\item[Ports:] 80, 443, 3389
\end{description}

\subsection{Implementation Steps}
\begin{enumerate}
	\item \textbf{Install IIS Role:}
	      \begin{lstlisting}[language=PowerShell]
Install-WindowsFeature -Name Web-Server -IncludeManagementTools
Install-WindowsFeature -Name Web-Asp-Net45, Web-Net-Ext45
Install-WindowsFeature -Name Web-Mgmt-Console
\end{lstlisting}
	\item \textbf{Configure IIS:}
	      \begin{lstlisting}[language=PowerShell]
# Create new website
New-Website -Name "MyWebsite" -Port 80 -PhysicalPath "C:\inetpub\MyWebsite" -ApplicationPool "DefaultAppPool"

# Create application pool
New-WebAppPool -Name "MyAppPool"
Set-ItemProperty IIS:\AppPools\MyAppPool -name "managedRuntimeVersion" -value "v4.0"
\end{lstlisting}
	\item \textbf{Install SSL Certificate:}
	      \begin{lstlisting}[language=PowerShell]
New-WebBinding -Name "MyWebsite" -Protocol "https" -Port 443 -SslFlags 0
\end{lstlisting}
\end{enumerate}

\section{Mail Server}

\subsection{AWS Configuration}
\begin{description}
	\item[Instance Type:] t3.medium
	\item[Ports:] 25 (SMTP), 110, 143, 587, 993, 995
	\item[Elastic IP:] Required
\end{description}

\begin{itemize}
	\item \textbf{Crucial:} AWS blocks port 25 by default. You must request removal via AWS Support.
	\item \textbf{Software Options:} hMailServer (Free) or Microsoft Exchange Server.
	\item \textbf{Alternative:} Use Amazon SES for better deliverability.
\end{itemize}

\section{Database Server}

\subsection{AWS Configuration}
\begin{description}
	\item[Instance Type:] r5.large or larger (Memory Optimized)
	\item[Storage:] Provisioned IOPS or io2
\end{description}

\subsection{Implementation Steps}

\subsubsection*{SQL Server}
\begin{lstlisting}[language=PowerShell]
# Silent installation example
Setup.exe /Q /ACTION=Install /FEATURES=SQLEngine /INSTANCENAME=MSSQLSERVER /SQLSYSADMINACCOUNTS="DOMAIN\SQLAdmins" /AGTSVCACCOUNT="NT AUTHORITY\SYSTEM" /SQLSVCACCOUNT="NT AUTHORITY\SYSTEM"
\end{lstlisting}

\begin{lstlisting}[language=SQL]
-- Enable remote connections
EXEC sys.sp_configure 'remote access', 1;
RECONFIGURE;

-- Configure max memory
EXEC sys.sp_configure 'max server memory (MB)', 8192;
RECONFIGURE;

-- Backup to S3
BACKUP DATABASE [MyDB] TO URL = 's3://my-bucket/backups/MyDB.bak'
\end{lstlisting}

\subsubsection*{PostgreSQL}
Edit \texttt{postgresql.conf} and \texttt{pg\_hba.conf}:
\begin{lstlisting}[language=Bash]
# postgresql.conf
listen_addresses = '*'
max_connections = 100
shared_buffers = 2GB

# pg_hba.conf
host all all 0.0.0.0/0 md5
\end{lstlisting}

\subsubsection*{MongoDB}
\begin{lstlisting}[language=Bash]
# mongod.cfg
net:
  port: 27017
  bindIp: 0.0.0.0
security:
  authorization: enabled
storage:
  dbPath: D:\MongoDB\data
\end{lstlisting}

\section{Backup Server}

\subsection{AWS Configuration}
\begin{itemize}
	\item \textbf{Instance:} t3.medium
	\item \textbf{Role:} IAM permissions for S3 and EBS snapshots.
\end{itemize}

\subsection{Implementation Steps}
\begin{enumerate}
	\item \textbf{Install Windows Server Backup:}
	      \begin{lstlisting}[language=PowerShell]
Install-WindowsFeature -Name Windows-Server-Backup -IncludeManagementTools
\end{lstlisting}
	\item \textbf{Configure Backup to S3 (Example):}
	      \begin{lstlisting}[language=PowerShell]
$Policy = New-WBPolicy
$Target = New-WBBackupTarget -VolumePath "D:"
Add-WBBackupTarget -Policy $Policy -Target $Target
Add-WBVolume -Policy $Policy -Volume (Get-WBVolume -VolumePath "C:")
Set-WBSchedule -Policy $Policy -Schedule 02:00
Set-WBPolicy -Policy $Policy
\end{lstlisting}
	\item Use AWS Backup for centralized management and S3 Glacier for archiving.
\end{enumerate}

\section{Load Balancing}

\subsection{Implementation Steps}
\begin{enumerate}
	\item Launch multiple identical servers in different availability zones.
	\item \textbf{Create Target Group:} Protocol HTTP/HTTPS, Health Check Path \texttt{/health}.
	\item \textbf{Create Application Load Balancer (ALB):} Add listener rules and register target group.
	\item \textbf{Session Persistence:} Enable sticky sessions if required.
\end{enumerate}

\section{Failover Cluster}

\subsection{AWS Configuration}
\begin{description}
	\item[Instance Type:] r5.xlarge or larger
	\item[Storage:] Shared storage via FSx for Windows or EBS Multi-Attach (io2).
\end{description}

\subsection{Implementation Steps}
\begin{enumerate}
	\item \textbf{Install Failover Clustering:}
	      \begin{lstlisting}[language=PowerShell]
Install-WindowsFeature -Name Failover-Clustering -IncludeManagementTools
\end{lstlisting}
	\item \textbf{Create Failover Cluster:}
	      \begin{lstlisting}[language=PowerShell]
# Validate
Test-Cluster -Node "Node1", "Node2"

# Create
New-Cluster -Name "MyCluster" -Node "Node1", "Node2" -StaticAddress "10.0.1.100" -NoStorage
\end{lstlisting}
	\item \textbf{Configure Quorum:}
	      \begin{lstlisting}[language=PowerShell]
Set-ClusterQuorum -NodeAndFileShareMajority "\\FSx\Witness"
\end{lstlisting}
	\item \textbf{Add Clustered Role (e.g., SQL):}
	      \begin{lstlisting}[language=PowerShell]
Add-ClusterServerRole -Name "SQL-Cluster" -Storage "Cluster Disk 1"
\end{lstlisting}
\end{enumerate}

\section{FTP Server}

\subsection{AWS Configuration}
\begin{description}
	\item[Ports:] 21, 20, 990, Passive Range (50000-50100)
	\item[Elastic IP:] Required
\end{description}

\subsection{Implementation Steps}
\begin{enumerate}
	\item \textbf{Install Role:}
	      \begin{lstlisting}[language=PowerShell]
Install-WindowsFeature -Name Web-Ftp-Server -IncludeManagementTools
Install-WindowsFeature -Name Web-Ftp-Service
\end{lstlisting}
	\item \textbf{Configure Site \& Passive Mode:}
	      \begin{lstlisting}[language=PowerShell]
New-WebFtpSite -Name "FTP Site" -Port 21 -PhysicalPath "D:\FTP"

# Passive Ports
Set-WebConfigurationProperty -Filter /system.ftpServer/firewallSupport -PSPath IIS:\ -Name lowDataChannelPort -Value 50000
Set-WebConfigurationProperty -Filter /system.ftpServer/firewallSupport -PSPath IIS:\ -Name highDataChannelPort -Value 50100
\end{lstlisting}
	\item \textbf{Enable FTPS (SSL):}
	      \begin{lstlisting}[language=PowerShell]
$cert = New-SelfSignedCertificate -DnsName "ftp.yourdomain.com" -CertStoreLocation cert:\LocalMachine\My
Set-WebConfigurationProperty -Filter /system.ftpServer/security/ssl -PSPath IIS:\ -Location "FTP Site" -Name serverCertHash -Value $cert.Thumbprint
Set-WebConfigurationProperty -Filter /system.ftpServer/security/ssl -PSPath IIS:\ -Location "FTP Site" -Name ssl128 -Value $true
\end{lstlisting}
\end{enumerate}

\section{Container (Docker)}

\subsection{AWS Configuration}
\begin{description}
	\item[OS:] Windows Server 2019/2022 with Containers
\end{description}

\subsection{Implementation Steps}
\begin{enumerate}
	\item \textbf{Install Docker:}
	      \begin{lstlisting}[language=PowerShell]
Install-Module -Name DockerMsftProvider -Repository PSGallery -Force
Install-Package -Name docker -ProviderName DockerMsftProvider -Force
Restart-Computer -Force
\end{lstlisting}
	\item \textbf{Create Dockerfile:}
	      \begin{lstlisting}[language=Bash]
FROM mcr.microsoft.com/dotnet/framework/aspnet:4.8
WORKDIR /inetpub/wwwroot
COPY ./app .
EXPOSE 80
\end{lstlisting}
	\item \textbf{Build and Run:}
	      \begin{lstlisting}[language=PowerShell]
docker build -t mywebapp:v1 .
docker run -d -p 80:80 --name webapp mywebapp:v1
\end{lstlisting}
	\item \textbf{Push to ECR:}
	      \begin{lstlisting}[language=PowerShell]
aws ecr get-login-password --region us-east-1 | docker login --username AWS --password-stdin ACCOUNT_ID.dkr.ecr.us-east-1.amazonaws.com
docker push ACCOUNT_ID.dkr.ecr.us-east-1.amazonaws.com/mywebapp:v1
\end{lstlisting}
\end{enumerate}

\section{Domain Controller}

\subsection{AWS Configuration}
\begin{description}
	\item[Instance Type:] t3.medium or larger
	\item[Ports:] 53, 88, 135, 139, 445, 389, 636, 3268, 3269, 49152-65535
	\item[Storage:] Minimum 50GB SSD
\end{description}

\subsection{Implementation Steps}

\subsubsection*{1. Initial Configuration}
Set static IP and rename the computer.
\begin{lstlisting}[language=PowerShell]
New-NetIPAddress -InterfaceAlias "Ethernet" -IPAddress 10.0.1.10 -PrefixLength 24 -DefaultGateway 10.0.1.1
Set-DnsClientServerAddress -InterfaceAlias "Ethernet" -ServerAddresses 127.0.0.1,8.8.8.8
Rename-Computer -NewName "DC01" -Restart
\end{lstlisting}

\subsubsection*{2. Install AD DS and Promote to DC}
\begin{lstlisting}[language=PowerShell]
Install-WindowsFeature -Name AD-Domain-Services -IncludeManagementTools
Import-Module ADDSDeployment

Install-ADDSForest `
    -DomainName "company.local" `
    -DomainNetbiosName "COMPANY" `
    -ForestMode "WinThreshold" `
    -DomainMode "WinThreshold" `
    -InstallDns:$true `
    -Force:$true
\end{lstlisting}

\subsubsection*{3. Post-Installation Config}
\begin{lstlisting}[language=PowerShell]
# Check status
Get-Service ADWS
Get-ADDomainController

# Create OUs
New-ADOrganizationalUnit -Name "Users" -Path "DC=company,DC=local"
New-ADOrganizationalUnit -Name "Computers" -Path "DC=company,DC=local"

# Enable Recycle Bin
Enable-ADOptionalFeature -Identity 'Recycle Bin Feature' -Scope ForestOrConfigurationSet -Target 'company.local' -Confirm:$false
\end{lstlisting}

\subsubsection*{4. Time Sync (PDC Emulator)}
\begin{lstlisting}[language=PowerShell]
w32tm /config /manualpeerlist:"time.windows.com,0x8" /syncfromflags:manual /reliable:yes /update
Restart-Service W32Time
\end{lstlisting}

\subsection{Security Best Practices}
\begin{itemize}
	\item Implement least privilege access.
	\item Use separate administrative accounts.
	\item Enable and monitor security logs.
	\item Regularly patch and update.
	\item Use strong password policies.
\end{itemize}

\end{document}
