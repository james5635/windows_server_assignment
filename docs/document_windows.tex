\documentclass[11pt, a4paper]{article}

% --- UNIVERSAL PREAMBLE BLOCK ---
\usepackage[a4paper, top=2.5cm, bottom=2.5cm, left=2cm, right=2cm]{geometry}
\usepackage{fontspec}
\usepackage[english, provide=*]{babel}

\babelprovide[import, onchar=ids fonts]{english}

% Set default/Latin font to Sans Serif in the main (rm) slot
\babelfont{rm}{Noto Sans}
\babelfont{sf}{Noto Sans}
\babelfont{tt}{Noto Sans Mono}

% Add because main language is not English (Standard Protocol) - though strictly English here, good for consistency
\usepackage{enumitem}
\setlist[itemize]{label=-}

% --- PACKAGES FOR THIS DOCUMENT ---
\usepackage{xcolor}
\usepackage{listings}
\usepackage{titlesec}
\usepackage{booktabs}

% Hyperref must be loaded last
\usepackage[hidelinks]{hyperref}
\hypersetup{
    colorlinks=true,
    linkcolor=blue,
    urlcolor=blue,
    pdftitle={Windows Server Deployment Guide on AWS Cloud}
}

% --- CODE LISTING CONFIGURATION ---
\definecolor{codegreen}{rgb}{0,0.6,0}
\definecolor{codegray}{rgb}{0.5,0.5,0.5}
\definecolor{codepurple}{rgb}{0.58,0,0.82}
\definecolor{backcolour}{rgb}{0.96,0.96,0.96}

\lstdefinestyle{mystyle}{
    backgroundcolor=\color{backcolour},
    commentstyle=\color{codegreen},
    keywordstyle=\color{magenta},
    numberstyle=\tiny\color{codegray},
    stringstyle=\color{codepurple},
    basicstyle=\ttfamily\footnotesize,
    breakatwhitespace=false,
    breaklines=true,
    captionpos=b,
    keepspaces=true,
    numbers=left,
    numbersep=5pt,
    showspaces=false,
    showstringspaces=false,
    showtabs=false,
    tabsize=2,
    frame=single
}

\lstset{style=mystyle}

% --- DOCUMENT START ---
\title{\textbf{Windows Server Deployment Guide on AWS Cloud}}
% \author{Sou Chanrojame}
\author{Sou Chanrojame, Orn Pheakdey, Long Neron, Then Sivthean, Le Sreyma}
\date{\today}

\begin{document}

\maketitle

\begin{abstract}
	This document provides a comprehensive step-by-step guide for deploying various Windows Server roles and services on the AWS Cloud infrastructure, including configuration details for EC2 instances, security groups, and storage optimization.
\end{abstract}


\tableofcontents
\newpage

\section*{Prerequisites}
\addcontentsline{toc}{section}{Prerequisites}

\subsection*{AWS Account Setup}
\begin{itemize}
	\item Active AWS account with appropriate permissions
	\item VPC configured with public and private subnets
	\item Security groups properly configured
	\item Key pairs created for RDP access
	\item IAM roles for EC2 instances
\end{itemize}

\subsection*{General Windows Server Launch Steps}
\begin{enumerate}
	\item Navigate to EC2 Dashboard in AWS Console
	\item Click ``Launch Instance''
	\item Select Windows Server AMI (2019/2022 recommended)
	\item Choose instance type based on workload
	\item Configure instance details (VPC, subnet, IAM role)
	\item Add storage as needed
	\item Configure security groups
	\item Review and launch with key pair
\end{enumerate}

% \hrule
\newpage

\section{File Server}

\subsection{AWS Configuration}
\textbf{Instance Type:} t3.medium or larger\\
\textbf{Storage:} EBS volumes with provisioned IOPS for performance\\
\textbf{Security Group Ports:} 445 (SMB), 139 (NetBIOS), 3389 (RDP)

\subsection{Implementation Steps}

\begin{enumerate}
	\item \textbf{Launch Windows Server EC2 Instance}
	      \begin{itemize}
		      \item Select Windows Server 2022 Datacenter
		      \item Attach additional EBS volumes for file storage
	      \end{itemize}

	\item \textbf{Install File Server Role}
	      \begin{lstlisting}[language=bash]
Install-WindowsFeature -Name FS-FileServer -IncludeManagementTools
Install-WindowsFeature -Name FS-DFS-Namespace, FS-DFS-Replication
\end{lstlisting}

	\item \textbf{Configure Storage}
	      \begin{itemize}
		      \item Initialize and format additional EBS volumes
		      \item Create shared folders
	      \end{itemize}
	      \begin{lstlisting}[language=bash]
New-SmbShare -Name "SharedFiles" -Path "D:\Shares" -FullAccess "Domain\Admins" -ReadAccess "Domain\Users"
\end{lstlisting}

	\item \textbf{Enable Shadow Copies}
	      \begin{lstlisting}[language=bash]
Enable-ComputerRestore -Drive "D:\"
vssadmin resize shadowstorage /for=D: /on=D: /maxsize=20%
\end{lstlisting}

	\item \textbf{Configure AWS Backup}
	      \begin{itemize}
		      \item Create backup plan for EBS volumes
		      \item Set retention policies
	      \end{itemize}
\end{enumerate}

\subsection{Best Practices}
\begin{itemize}
	\item Use AWS Storage Gateway for hybrid scenarios
	\item Implement Amazon FSx for Windows File Server for managed solution
	\item Enable encryption at rest using AWS KMS
	\item Configure NTFS permissions and share permissions
\end{itemize}

\newpage

\section{Proxy Server (Caching, Control Access)}

\subsection{AWS Configuration}
\textbf{Instance Type:} t3.medium\\
\textbf{Security Group Ports:} 8080, 3128 (proxy), 3389 (RDP)

\subsection{Implementation Steps}

\begin{enumerate}
	\item \textbf{Launch Windows Server Instance}

	\item \textbf{Install Proxy Server Software}
	      \begin{itemize}
		      \item \textbf{Option A: Windows Server with WinGate}\\
		            Download and install WinGate; Configure proxy settings.
		      \item \textbf{Option B: Squid for Windows}\\
		            Download Squid for Windows; Install and configure \texttt{squid.conf}.
	      \end{itemize}

	\item \textbf{Configure Proxy Settings}
	      \begin{lstlisting}[language=bash]
# Example configuration for basic proxy
netsh winhttp set proxy proxy-server="localhost:8080" bypass-list="*.local"
\end{lstlisting}

	\item \textbf{Set Up Caching}
	      \begin{itemize}
		      \item Configure cache directory on separate EBS volume
		      \item Set cache size limits
		      \item Define cache policies
	      \end{itemize}

	\item \textbf{Access Control}
	      \begin{itemize}
		      \item Configure authentication (AD integration)
		      \item Set up URL filtering rules
		      \item Implement blacklists/whitelists
	      \end{itemize}

	\item \textbf{Configure AWS Security Group}
	      \begin{itemize}
		      \item Allow inbound traffic on proxy port from specific CIDR blocks
		      \item Restrict outbound traffic as needed
	      \end{itemize}
\end{enumerate}

\subsection{Best Practices}
\begin{itemize}
	\item Use AWS Network Firewall for additional security
	\item Consider AWS Global Accelerator for multiple regions
	\item Monitor with CloudWatch metrics
	\item Use ALB/NLB for proxy clustering
\end{itemize}

\newpage

\section{DNS Server}

\subsection{AWS Configuration}
\textbf{Instance Type:} t3.small\\
\textbf{Security Group Ports:} 53 (TCP/UDP), 3389 (RDP)

\subsection{Implementation Steps}

\begin{enumerate}
	\item \textbf{Launch Windows Server Instance}
	      \begin{itemize}
		      \item Place in private subnet for internal DNS
	      \end{itemize}

	\item \textbf{Install DNS Server Role}
	      \begin{lstlisting}[language=bash]
Install-WindowsFeature -Name DNS -IncludeManagementTools
\end{lstlisting}

	\item \textbf{Configure DNS Zones}
	      \begin{lstlisting}[language=bash]
# Create Primary Zone
Add-DnsServerPrimaryZone -Name "yourdomain.local" -ReplicationScope "Forest" -PassThru

# Create Reverse Lookup Zone
Add-DnsServerPrimaryZone -NetworkID "10.0.0.0/16" -ReplicationScope "Forest"
\end{lstlisting}

	\item \textbf{Configure Forwarders}
	      \begin{lstlisting}[language=bash]
# Use AWS DNS or external DNS
Add-DnsServerForwarder -IPAddress "8.8.8.8", "8.8.4.4"
\end{lstlisting}

	\item \textbf{Integrate with AWS Route 53}
	      \begin{itemize}
		      \item Create Route 53 Resolver endpoints
		      \item Configure conditional forwarding for AWS resources
	      \end{itemize}

	\item \textbf{Configure DHCP Option Sets}
	      \begin{itemize}
		      \item Update VPC DHCP options to point to DNS server
	      \end{itemize}
\end{enumerate}

\subsection{Best Practices}
\begin{itemize}
	\item Deploy multiple DNS servers for redundancy
	\item Use Route 53 for public DNS records
	\item Enable DNS logging and monitoring
	\item Implement DNSSEC for security
\end{itemize}

\newpage

\section{DHCP Server}

\subsection{AWS Configuration}
\textbf{Instance Type:} t3.small\\
\textbf{Note:} AWS VPC provides DHCP by default; custom DHCP server is optional.

\subsection{Implementation Steps}

\begin{enumerate}
	\item \textbf{Launch Windows Server Instance}

	\item \textbf{Install DHCP Server Role}
	      \begin{lstlisting}[language=bash]
Install-WindowsFeature -Name DHCP -IncludeManagementTools
Add-DhcpServerInDC -DnsName "dhcp.yourdomain.local"
\end{lstlisting}

	\item \textbf{Configure DHCP Scope}
	      \begin{lstlisting}[language=bash]
Add-DhcpServerv4Scope -Name "Internal Network" -StartRange 10.0.1.100 -EndRange 10.0.1.200 -SubnetMask 255.255.255.0

Set-DhcpServerv4OptionValue -ScopeId 10.0.1.0 -Router 10.0.1.1
Set-DhcpServerv4OptionValue -ScopeId 10.0.1.0 -DnsServer 10.0.1.10
\end{lstlisting}

	\item \textbf{Configure Reservations}
	      \begin{lstlisting}[language=bash]
Add-DhcpServerv4Reservation -ScopeId 10.0.1.0 -IPAddress 10.0.1.50 -ClientId "00-11-22-33-44-55" -Description "Print Server"
\end{lstlisting}

	\item \textbf{Authorize DHCP Server}
	      \begin{lstlisting}[language=bash]
Add-DhcpServerInDC -DnsName "dhcp.yourdomain.local" -IPAddress 10.0.1.10
\end{lstlisting}
\end{enumerate}

\subsection{Best Practices}
\begin{itemize}
	\item Consider using AWS-provided DHCP for simplicity
	\item Deploy DHCP failover for redundancy
	\item Use DHCP policies for different device types
	\item Monitor DHCP lease utilization
\end{itemize}

\newpage

\section{VPN Server}

\subsection{AWS Configuration}
\textbf{Instance Type:} t3.small to t3.medium\\
\textbf{Security Group Ports:} 1723 (PPTP), 1701 (L2TP), 500/4500 (IPSec), 443 (SSTP)\\
\textbf{Elastic IP:} Required for consistent endpoint

\subsection{Implementation Steps}

\begin{enumerate}
	\item \textbf{Launch Windows Server Instance with Elastic IP}

	\item \textbf{Install Remote Access Role}
	      \begin{lstlisting}[language=bash]
Install-WindowsFeature -Name RemoteAccess -IncludeManagementTools
Install-WindowsFeature -Name DirectAccess-VPN -IncludeManagementTools
Install-WindowsFeature -Name Routing -IncludeManagementTools
\end{lstlisting}

	\item \textbf{Configure VPN Server}
	      \begin{lstlisting}[language=bash]
Install-RemoteAccess -VpnType Vpn
\end{lstlisting}

	\item \textbf{Configure VPN Protocols}
	      \begin{itemize}
		      \item Enable SSTP, L2TP/IPSec, or IKEv2
		      \item Configure authentication methods (RADIUS, certificates)
	      \end{itemize}

	\item \textbf{Set Up IP Address Assignment}
	      \begin{lstlisting}[language=bash]
Set-VpnServerConfiguration -TunnelType SSTP -PassThru
\end{lstlisting}

	\item \textbf{Configure Routing}
	      \begin{itemize}
		      \item Enable NAT for VPN clients
		      \item Configure routing tables
	      \end{itemize}
\end{enumerate}

\subsection{Alternative: AWS Client VPN}
Consider using AWS Client VPN for managed VPN service with better scalability and integration.

\subsection{Best Practices}
\begin{itemize}
	\item Use certificate-based authentication
	\item Integrate with AWS Directory Service
	\item Monitor connections with CloudWatch
	\item Consider AWS Site-to-Site VPN for office connectivity
\end{itemize}

\newpage

\section{Terminal Server (Thin Clients)}

\subsection{AWS Configuration}
\textbf{Instance Type:} t3.xlarge or larger (based on user count)\\
\textbf{Security Group Ports:} 3389 (RDP), 3391 (RD Gateway)

\subsection{Implementation Steps}

\begin{enumerate}
	\item \textbf{Launch Windows Server Instance}
	      \begin{itemize}
		      \item Size appropriately for concurrent users (2 vCPU + 4GB RAM per 5-10 users)
	      \end{itemize}

	\item \textbf{Install RDS Roles}
	      \begin{lstlisting}[language=bash]
Install-WindowsFeature -Name RDS-RD-Server -IncludeManagementTools
Install-WindowsFeature -Name RDS-Connection-Broker -IncludeManagementTools
Install-WindowsFeature -Name RDS-Web-Access -IncludeManagementTools
Install-WindowsFeature -Name RDS-Gateway -IncludeManagementTools
Install-WindowsFeature -Name RDS-Licensing -IncludeManagementTools
\end{lstlisting}

	\item \textbf{Configure RDS Deployment}
	      \begin{itemize}
		      \item Use Server Manager to create RDS deployment
		      \item Add RD Session Host
		      \item Configure RD Gateway for external access
	      \end{itemize}

	\item \textbf{Install RDS CALs}
	      \begin{itemize}
		      \item Install RDS License Server
		      \item Activate and install Client Access Licenses
	      \end{itemize}

	\item \textbf{Configure Session Collections}
	      \begin{lstlisting}[language=bash]
New-RDSessionCollection -CollectionName "Production" -SessionHost "rdsh01.yourdomain.local" -ConnectionBroker "rdcb.yourdomain.local"
\end{lstlisting}

	\item \textbf{Set Up RemoteApp}
	      \begin{lstlisting}[language=bash]
New-RDRemoteApp -CollectionName "Production" -DisplayName "Microsoft Word" -FilePath "C:\Program Files\Microsoft Office\root\Office16\WINWORD.EXE"
\end{lstlisting}
\end{enumerate}

\subsection{Best Practices}
\begin{itemize}
	\item Use RD Gateway with SSL certificates
	\item Deploy multiple Session Hosts with load balancing
	\item Use FSLogix for user profile management
	\item Store user data on separate file server
	\item Consider Amazon WorkSpaces for managed VDI
\end{itemize}

\newpage

\section{Web Server}

\subsection{AWS Configuration}
\textbf{Instance Type:} t3.medium\\
\textbf{Security Group Ports:} 80 (HTTP), 443 (HTTPS), 3389 (RDP)\\
\textbf{Load Balancer:} Application Load Balancer recommended

\subsection{Implementation Steps}

\begin{enumerate}
	\item \textbf{Launch Windows Server Instance}

	\item \textbf{Install IIS Role}
	      \begin{lstlisting}[language=bash]
Install-WindowsFeature -Name Web-Server -IncludeManagementTools
Install-WindowsFeature -Name Web-Asp-Net45, Web-Net-Ext45
Install-WindowsFeature -Name Web-Mgmt-Console
\end{lstlisting}

	\item \textbf{Configure IIS}
	      \begin{lstlisting}[language=bash]
# Create new website
New-Website -Name "MyWebsite" -Port 80 -PhysicalPath "C:\inetpub\MyWebsite" -ApplicationPool "DefaultAppPool"

# Create application pool
New-WebAppPool -Name "MyAppPool"
Set-ItemProperty IIS:\AppPools\MyAppPool -name "managedRuntimeVersion" -value "v4.0"
\end{lstlisting}

	\item \textbf{Install SSL Certificate}
	      \begin{itemize}
		      \item Request certificate from AWS Certificate Manager
		      \item Import certificate to IIS
	      \end{itemize}
	      \begin{lstlisting}[language=bash]
New-WebBinding -Name "MyWebsite" -Protocol "https" -Port 443 -SslFlags 0
\end{lstlisting}

	\item \textbf{Configure Application Settings}
	      \begin{itemize}
		      \item Set up .NET Framework or .NET Core
		      \item Configure connection strings
		      \item Set permissions for application folders
	      \end{itemize}

	\item \textbf{Set Up Application Load Balancer}
	      \begin{itemize}
		      \item Create target group with health checks
		      \item Register EC2 instances
		      \item Configure listener rules
	      \end{itemize}
\end{enumerate}

\subsection{Best Practices}
\begin{itemize}
	\item Use AWS Certificate Manager for SSL certificates
	\item Enable CloudFront for CDN
	\item Configure auto-scaling for traffic spikes
	\item Use Amazon RDS instead of local database
	\item Enable AWS WAF for security
\end{itemize}

\newpage

\section{Mail Server}

\subsection{AWS Configuration}
\textbf{Instance Type:} t3.medium\\
\textbf{Security Group Ports:} 25 (SMTP), 110 (POP3), 143 (IMAP), 587 (Submission), 993 (IMAPS), 995 (POP3S)\\
\textbf{Elastic IP:} Required\\
\textbf{Note:} AWS blocks port 25 by default; request removal.

\subsection{Implementation Steps}

\begin{enumerate}
	\item \textbf{Request Port 25 Unblocking}
	      \begin{itemize}
		      \item Submit request to AWS Support
		      \item Provide reverse DNS setup
	      \end{itemize}

	\item \textbf{Launch Windows Server Instance with Elastic IP}

	\item \textbf{Install SMTP Server}
	      \begin{lstlisting}[language=bash]
Install-WindowsFeature -Name SMTP-Server -IncludeManagementTools
\end{lstlisting}

	\item \textbf{Install Third-Party Mail Server}
	      \begin{itemize}
		      \item \textbf{Option A: hMailServer (Free)}\\
		            Download and install hMailServer; Configure domains and accounts; Set up SSL/TLS certificates.
		      \item \textbf{Option B: Microsoft Exchange Server}\\
		            More complex but full-featured; Install prerequisites; Install Exchange Server; Configure mailbox databases.
	      \end{itemize}

	\item \textbf{Configure DNS Records}
	      \begin{itemize}
		      \item MX records pointing to Elastic IP
		      \item SPF, DKIM, and DMARC records
		      \item Reverse DNS (PTR) record
	      \end{itemize}

	\item \textbf{Configure Security}
	      \begin{itemize}
		      \item Enable spam filtering
		      \item Configure antivirus scanning
		      \item Set up SSL/TLS encryption
		      \item Configure relay restrictions
	      \end{itemize}
\end{enumerate}

\subsection{Alternative: Amazon SES}
Consider using Amazon Simple Email Service (SES) for sending emails, which provides better deliverability and doesn't require managing infrastructure.

\subsection{Best Practices}
\begin{itemize}
	\item Use Amazon WorkMail for managed email service
	\item Implement proper email security (SPF, DKIM, DMARC)
	\item Configure backup MX records
	\item Monitor email queues and logs
	\item Use SES for transactional emails
\end{itemize}

\newpage

\section{Database Server}

\subsection{AWS Configuration}
\textbf{Instance Type:} r5.large or larger (memory-optimized)\\
\textbf{Storage:} EBS with provisioned IOPS or io2\\
\textbf{Security Group Ports:}
\begin{itemize}
	\item MongoDB: 27017
	\item Oracle: 1521
	\item SQL Server: 1433
	\item PostgreSQL: 5432
\end{itemize}

\subsection{Implementation Steps}

\subsubsection{SQL Server}
\begin{enumerate}
	\item \textbf{Launch Windows Server Instance}
	      \begin{itemize}
		      \item Use memory-optimized instance type
		      \item Attach high-performance EBS volumes
	      \end{itemize}

	\item \textbf{Install SQL Server}
	      \begin{itemize}
		      \item Download SQL Server (Developer/Standard/Enterprise)
		      \item Run setup.exe
	      \end{itemize}
	      \begin{lstlisting}[language=bash]
# Silent installation example
Setup.exe /Q /ACTION=Install /FEATURES=SQLEngine /INSTANCENAME=MSSQLSERVER /SQLSYSADMINACCOUNTS="DOMAIN\SQLAdmins" /AGTSVCACCOUNT="NT AUTHORITY\SYSTEM" /SQLSVCACCOUNT="NT AUTHORITY\SYSTEM"
\end{lstlisting}

	\item \textbf{Configure SQL Server}
	      \begin{lstlisting}[language=sql]
-- Enable remote connections
EXEC sys.sp_configure 'remote access', 1;
RECONFIGURE;

-- Configure max memory
EXEC sys.sp_configure 'max server memory (MB)', 8192;
RECONFIGURE;
\end{lstlisting}

	\item \textbf{Set Up Backups}
	      \begin{itemize}
		      \item Configure SQL Server backup to S3
		      \item Use SQL Server native backup to S3
	      \end{itemize}
	      \begin{lstlisting}[language=sql]
BACKUP DATABASE [MyDB] TO URL = 's3://my-bucket/backups/MyDB.bak'
\end{lstlisting}
\end{enumerate}

\subsubsection{PostgreSQL}
\begin{enumerate}
	\item \textbf{Install PostgreSQL}
	      \begin{itemize}
		      \item Download PostgreSQL installer for Windows
		      \item Run installation wizard
	      \end{itemize}

	\item \textbf{Configure PostgreSQL}
	      \begin{lstlisting}[language=bash]
# Edit postgresql.conf
listen_addresses = '*'
max_connections = 100
shared_buffers = 2GB

# Edit pg_hba.conf for authentication
host all all 0.0.0.0/0 md5
\end{lstlisting}

	\item \textbf{Create Database}
	      \begin{lstlisting}[language=sql]
CREATE DATABASE myapp;
CREATE USER appuser WITH ENCRYPTED PASSWORD 'password';
GRANT ALL PRIVILEGES ON DATABASE myapp TO appuser;
\end{lstlisting}
\end{enumerate}

\subsubsection{MongoDB}
\begin{enumerate}
	\item \textbf{Install MongoDB}
	      \begin{itemize}
		      \item Download MongoDB Community Server for Windows
		      \item Install as Windows Service
	      \end{itemize}

	\item \textbf{Configure MongoDB}
	      \begin{lstlisting}[language=bash]
# Edit mongod.cfg
net:
  port: 27017
  bindIp: 0.0.0.0
security:
  authorization: enabled
storage:
  dbPath: D:\MongoDB\data
\end{lstlisting}

	\item \textbf{Create Admin User}
	      \begin{lstlisting}[language=bash]
use admin
db.createUser({
  user: "admin",
  pwd: "password",
  roles: [ { role: "root", db: "admin" } ]
})
\end{lstlisting}
\end{enumerate}

\subsection{Alternative: Amazon RDS}
Consider using Amazon RDS for SQL Server, PostgreSQL, or Oracle for fully managed database service with automated backups, patching, and high availability.

\subsection{Best Practices}
\begin{itemize}
	\item Use Amazon RDS for managed database services
	\item Enable automated backups
	\item Use Multi-AZ deployments for high availability
	\item Store database files on separate EBS volumes
	\item Enable encryption at rest
	\item Use IAM database authentication where possible
	\item Monitor with CloudWatch and Performance Insights
\end{itemize}

\newpage

\section{Backup Server}

\subsection{AWS Configuration}
\textbf{Instance Type:} t3.medium\\
\textbf{Storage:} Large EBS volumes or S3 integration\\
\textbf{IAM Role:} Permissions for S3, EBS snapshots

\subsection{Implementation Steps}

\begin{enumerate}
	\item \textbf{Launch Windows Server Instance}

	\item \textbf{Install Windows Server Backup}
	      \begin{lstlisting}[language=bash]
Install-WindowsFeature -Name Windows-Server-Backup -IncludeManagementTools
\end{lstlisting}

	\item \textbf{Configure AWS Backup}
	      \begin{itemize}
		      \item Set up AWS Backup service
		      \item Create backup plans
		      \item Assign resources to backup plans
	      \end{itemize}

	\item \textbf{Install Third-Party Backup Software}
	      \begin{itemize}
		      \item \textbf{Option A: Veeam Backup}\\
		            Download Veeam Backup \& Replication; Install and configure; Set up backup jobs to S3.
		      \item \textbf{Option B: Windows Server Backup to S3}
	      \end{itemize}
	      \begin{lstlisting}[language=bash]
# Create backup policy
$Policy = New-WBPolicy
$Target = New-WBBackupTarget -VolumePath "D:"
Add-WBBackupTarget -Policy $Policy -Target $Target
Add-WBVolume -Policy $Policy -Volume (Get-WBVolume -VolumePath "C:")
Set-WBSchedule -Policy $Policy -Schedule 02:00
Set-WBPolicy -Policy $Policy
\end{lstlisting}

	\item \textbf{Configure S3 Lifecycle Policies}
	      \begin{itemize}
		      \item Transition to S3 Glacier for long-term retention
		      \item Set expiration policies
	      \end{itemize}

	\item \textbf{Set Up EBS Snapshot Automation}
	      \begin{lstlisting}[language=bash]
# Using AWS PowerShell
New-EC2Snapshot -VolumeId vol-12345678 -Description "Daily Backup"
\end{lstlisting}
\end{enumerate}

\subsection{Best Practices}
\begin{itemize}
	\item Use AWS Backup for centralized backup management
	\item Store backups in S3 with versioning enabled
	\item Implement 3-2-1 backup strategy
	\item Test backup restoration regularly
	\item Use S3 Glacier for long-term archival
	\item Enable cross-region backup replication
\end{itemize}

\newpage

\section{Load Balancing}

\subsection{AWS Configuration}
\textbf{Service:} Application Load Balancer (ALB) or Network Load Balancer (NLB)\\
\textbf{Target Group:} Multiple Windows Server instances

\subsection{Implementation Steps}

\begin{enumerate}
	\item \textbf{Launch Multiple Windows Server Instances}
	      \begin{itemize}
		      \item Deploy identical servers in different availability zones
		      \item Install and configure web application on all instances
	      \end{itemize}

	\item \textbf{Create Target Group}
	      \begin{itemize}
		      \item Navigate to EC2 > Target Groups
		      \item Create target group with health check settings
	      \end{itemize}
	      \begin{verbatim}
Protocol: HTTP/HTTPS
Port: 80/443
Health Check Path: /health
Health Check Interval: 30 seconds
Healthy Threshold: 2
Unhealthy Threshold: 2
\end{verbatim}

	\item \textbf{Create Application Load Balancer}
	      \begin{itemize}
		      \item Choose ALB for HTTP/HTTPS traffic
		      \item Select availability zones
		      \item Configure security groups
		      \item Add listener rules
		      \item Register target group
	      \end{itemize}

	\item \textbf{Configure Session Persistence}
	      \begin{itemize}
		      \item Enable sticky sessions if needed
		      \item Configure duration
	      \end{itemize}

	\item \textbf{Set Up Auto Scaling Group}
	      \begin{lstlisting}[language=bash]
# Using AWS CLI or CloudFormation
# Define launch template
# Create auto-scaling group
# Configure scaling policies
\end{lstlisting}

	\item \textbf{Install and Configure IIS ARR (Alternative)}
	      \begin{lstlisting}[language=bash]
# For Windows-based load balancing
Install-WindowsFeature Web-Server -IncludeManagementTools
# Install Application Request Routing
# Configure server farms
\end{lstlisting}
\end{enumerate}

\subsection{Best Practices}
\begin{itemize}
	\item Use ALB for HTTP/HTTPS traffic
	\item Use NLB for TCP/UDP traffic or ultra-low latency
	\item Deploy instances across multiple availability zones
	\item Configure proper health checks
	\item Enable access logs for troubleshooting
	\item Use CloudWatch for monitoring
	\item Implement auto-scaling based on metrics
\end{itemize}

\newpage

\section{Failover Cluster}

\subsection{AWS Configuration}
\textbf{Instance Type:} r5.xlarge or larger\\
\textbf{Storage:} Shared storage using FSx for Windows or S3\\
\textbf{Network:} Placement groups for low latency\\
\textbf{Security Group:} Allow cluster communication ports

\subsection{Implementation Steps}

\begin{enumerate}
	\item \textbf{Launch Multiple Windows Server Instances}
	      \begin{itemize}
		      \item Deploy in same VPC, different availability zones
		      \item Use placement group for low latency
	      \end{itemize}

	\item \textbf{Install Failover Clustering Feature}
	      \begin{lstlisting}[language=bash]
Install-WindowsFeature -Name Failover-Clustering -IncludeManagementTools
\end{lstlisting}

	\item \textbf{Configure Shared Storage}
	      \begin{itemize}
		      \item \textbf{Option A: Amazon FSx for Windows File Server}\\
		            Create FSx file system; Mount on all cluster nodes.
		      \item \textbf{Option B: EBS Multi-Attach (io2 only)}\\
		            Attach same EBS volume to multiple instances; Initialize as cluster shared volume.
	      \end{itemize}

	\item \textbf{Create Failover Cluster}
	      \begin{lstlisting}[language=bash]
# Validate cluster configuration
Test-Cluster -Node "Node1", "Node2"

# Create cluster
New-Cluster -Name "MyCluster" -Node "Node1", "Node2" -StaticAddress "10.0.1.100" -NoStorage
\end{lstlisting}

	\item \textbf{Configure Cluster Quorum}
	      \begin{lstlisting}[language=bash]
Set-ClusterQuorum -NodeAndFileShareMajority "\\FSx\Witness"
\end{lstlisting}

	\item \textbf{Add Clustered Role}
	      \begin{lstlisting}[language=bash]
# For SQL Server
Add-ClusterServerRole -Name "SQL-Cluster" -Storage "Cluster Disk 1"
\end{lstlisting}

	\item \textbf{Configure Secondary Private IP}
	      \begin{itemize}
		      \item Assign secondary private IP to ENI
		      \item Configure in cluster as virtual IP
	      \end{itemize}
\end{enumerate}

\subsection{Common Cluster Types in AWS}
\begin{description}
	\item[SQL Server Failover Cluster] Use FSx for shared storage; Configure SQL Server on cluster nodes; Set up availability group for database replication.
	\item[File Server Cluster] Use FSx or S3 for storage; Configure highly available file shares.
\end{description}

\subsection{Best Practices}
\begin{itemize}
	\item Use Amazon FSx for Windows File Server for shared storage
	\item Deploy cluster nodes in different availability zones
	\item Use Elastic IP or Network Load Balancer for client access
	\item Monitor cluster health with CloudWatch
	\item Regular testing of failover scenarios
	\item Consider Amazon RDS Multi-AZ for database clustering
\end{itemize}

\newpage

\section{FTP Server}

\subsection{AWS Configuration}
\textbf{Instance Type:} t3.small to t3.medium\\
\textbf{Security Group Ports:} 21 (FTP Control), 20 (FTP Data), 990 (FTPS), Range for Passive Mode (e.g., 50000-50100)\\
\textbf{Elastic IP:} Required for consistent access

\subsection{Implementation Steps}

\begin{enumerate}
	\item \textbf{Launch Windows Server Instance with Elastic IP}

	\item \textbf{Install FTP Server Role}
	      \begin{lstlisting}[language=bash]
Install-WindowsFeature -Name Web-Ftp-Server -IncludeManagementTools
Install-WindowsFeature -Name Web-Ftp-Service
\end{lstlisting}

	\item \textbf{Configure FTP Site}
	      \begin{lstlisting}[language=bash]
# Create FTP site
New-WebFtpSite -Name "FTP Site" -Port 21 -PhysicalPath "D:\FTP"

# Configure authentication
Set-WebConfigurationProperty -Filter /system.ftpServer/security/authentication/basicAuthentication -PSPath IIS:\ -Location "FTP Site" -Name enabled -Value $true
\end{lstlisting}

	\item \textbf{Configure Passive Mode}
	      \begin{lstlisting}[language=bash]
# Set passive port range
Set-WebConfigurationProperty -Filter /system.ftpServer/firewallSupport -PSPath IIS:\ -Name lowDataChannelPort -Value 50000
Set-WebConfigurationProperty -Filter /system.ftpServer/firewallSupport -PSPath IIS:\ -Name highDataChannelPort -Value 50100

# Set external IP
Set-WebConfigurationProperty -Filter /system.ftpServer/firewallSupport -PSPath IIS:\ -Name externalIp4Address -Value "YOUR_ELASTIC_IP"
\end{lstlisting}

	\item \textbf{Enable FTPS (FTP over SSL)}
	      \begin{lstlisting}[language=bash]
# Import SSL certificate
$cert = New-SelfSignedCertificate -DnsName "ftp.yourdomain.com" -CertStoreLocation cert:\LocalMachine\My

# Bind certificate to FTP site
Set-WebConfigurationProperty -Filter /system.ftpServer/security/ssl -PSPath IIS:\ -Location "FTP Site" -Name serverCertHash -Value $cert.Thumbprint
Set-WebConfigurationProperty -Filter /system.ftpServer/security/ssl -PSPath IIS:\ -Location "FTP Site" -Name ssl128 -Value $true
\end{lstlisting}

	\item \textbf{Configure User Access}
	      \begin{lstlisting}[language=bash]
# Create FTP user
New-LocalUser -Name "ftpuser" -Password (ConvertTo-SecureString "Password123!" -AsPlainText -Force)

# Set folder permissions
$acl = Get-Acl "D:\FTP"
$permission = "ftpuser","FullControl","Allow"
$accessRule = New-Object System.Security.AccessControl.FileSystemAccessRule $permission
$acl.SetAccessRule($accessRule)
Set-Acl "D:\FTP" $acl
\end{lstlisting}

	\item \textbf{Configure Security Group}
	      \begin{itemize}
		      \item Allow port 21 (control)
		      \item Allow passive port range (50000-50100)
		      \item Restrict source IPs if possible
	      \end{itemize}
\end{enumerate}

\subsection{Alternative: AWS Transfer Family}
Consider using AWS Transfer Family (SFTP, FTPS, FTP) for fully managed file transfer service with S3 backend.

\subsection{Best Practices}
\begin{itemize}
	\item Use FTPS or SFTP instead of plain FTP
	\item Use AWS Transfer Family for managed solution
	\item Store files on S3 via AWS Transfer Family
	\item Limit source IP addresses in security groups
	\item Use separate EBS volume for FTP data
	\item Monitor with CloudWatch logs
	\item Regular security audits
\end{itemize}

\newpage

\section{Container (Docker)}

\subsection{AWS Configuration}
\textbf{Instance Type:} t3.medium or larger\\
\textbf{Operating System:} Windows Server 2019/2022 with Containers\\
\textbf{Security Group Ports:} Custom ports based on containerized applications

\subsection{Implementation Steps}

\begin{enumerate}
	\item \textbf{Launch Windows Server Instance}
	      \begin{itemize}
		      \item Choose Windows Server 2019/2022
		      \item Select version with container support
	      \end{itemize}

	\item \textbf{Install Docker}
	      \begin{lstlisting}[language=bash]
# Install Docker provider
Install-Module -Name DockerMsftProvider -Repository PSGallery -Force

# Install Docker
Install-Package -Name docker -ProviderName DockerMsftProvider -Force

# Restart computer
Restart-Computer -Force
\end{lstlisting}

	\item \textbf{Verify Docker Installation}
	      \begin{lstlisting}[language=bash]
docker version
docker info
\end{lstlisting}

	\item \textbf{Pull Windows Container Images}
	      \begin{lstlisting}[language=bash]
# Pull Windows Server Core base image
docker pull mcr.microsoft.com/windows/servercore:ltsc2022

# Pull .NET Framework image
docker pull mcr.microsoft.com/dotnet/framework/aspnet:4.8
\end{lstlisting}

	\item \textbf{Create Dockerfile}
	      \begin{lstlisting}[language=bash]
FROM mcr.microsoft.com/dotnet/framework/aspnet:4.8
WORKDIR /inetpub/wwwroot
COPY ./app .
EXPOSE 80
\end{lstlisting}

	\item \textbf{Build and Run Container}
	      \begin{lstlisting}[language=bash]
# Build image
docker build -t mywebapp:v1 .

# Run container
docker run -d -p 80:80 --name webapp mywebapp:v1

# View running containers
docker ps
\end{lstlisting}

	\item \textbf{Push to Amazon ECR}
	      \begin{lstlisting}[language=bash]
# Authenticate to ECR
aws ecr get-login-password --region us-east-1 | docker login --username AWS --password-stdin ACCOUNT_ID.dkr.ecr.us-east-1.amazonaws.com

# Tag image
docker tag mywebapp:v1 ACCOUNT_ID.dkr.ecr.us-east-1.amazonaws.com/mywebapp:v1

# Push image
docker push ACCOUNT_ID.dkr.ecr.us-east-1.amazonaws.com/mywebapp:v1
\end{lstlisting}
\end{enumerate}

\subsection{Alternative: Amazon ECS for Windows Containers}
Use Amazon Elastic Container Service (ECS) or Amazon Elastic Kubernetes Service (EKS) with Windows support for orchestrated container deployments.

\subsubsection{ECS Windows Container Setup}
\begin{enumerate}
	\item \textbf{Create ECS Cluster}
	      \begin{itemize}
		      \item Choose EC2 launch type with Windows AMI
		      \item Or use Fargate for Windows (when available)
	      \end{itemize}

	\item \textbf{Create Task Definition}
	      \begin{lstlisting}[language=bash]
{
  "family": "windows-webapp",
  "containerDefinitions": [
    {
      "name": "webapp",
      "image": "ACCOUNT_ID.dkr.ecr.us-east-1.amazonaws.com/mywebapp:v1",
      "memory": 2048,
      "cpu": 1024,
      "portMappings": [
        {
          "containerPort": 80,
          "protocol": "tcp"
        }
      ]
    }
  ],
  "requiresCompatibilities": ["EC2"],
  "networkMode": "awsvpc",
  "runtimePlatform": {
    "operatingSystemFamily": "WINDOWS_SERVER_2022_CORE"
  }
}
\end{lstlisting}

	\item \textbf{Create ECS Service}
	      \begin{itemize}
		      \item Deploy task definition
		      \item Configure load balancer
		      \item Set desired task count
	      \end{itemize}
\end{enumerate}

\subsection{Best Practices}
\begin{itemize}
	\item Use Amazon ECS or EKS for production container orchestration
	\item Store images in Amazon ECR
	\item Use Windows Server Core or Nano Server base images
	\item Implement CI/CD with AWS CodePipeline
	\item Monitor containers with CloudWatch Container Insights
	\item Use Task roles for AWS service access
	\item Regular image security scanning
\end{itemize}

\newpage

\section{Domain Controller}

\subsection{AWS Configuration}
\textbf{Instance Type:} t3.medium or larger\\
\textbf{Security Group Ports:}
\begin{itemize}
	\item 53 (DNS TCP/UDP)
	\item 88 (Kerberos)
	\item 135 (RPC)
	\item 139, 445 (SMB)
	\item 389, 636 (LDAP, LDAPS)
	\item 3268, 3269 (Global Catalog)
	\item 49152-65535 (Dynamic RPC)
\end{itemize}
\textbf{Storage:} Minimum 50GB SSD\\
\textbf{Operating System:} Windows Server 2019/2022

\subsection{Installation Steps}

\begin{enumerate}
	\item \textbf{Prepare the Server}
	      \begin{itemize}
		      \item Set a static IP address in Windows network settings
		      \item Configure DNS to point to itself (127.0.0.1) and a secondary DNS
		      \item Rename the server with a descriptive hostname
		      \item Ensure the system is fully updated
	      \end{itemize}

	\item \textbf{Install Active Directory Domain Services}
	      \begin{itemize}
		      \item Open Server Manager
		      \item Click ``Add roles and features''
		      \item Select ``Active Directory Domain Services'' role
		      \item Include management tools when prompted
		      \item Complete the installation wizard
	      \end{itemize}

	\item \textbf{Promote to Domain Controller}
	      \begin{itemize}
		      \item Click the notification flag in Server Manager
		      \item Select ``Promote this server to a domain controller''
		      \item Choose ``Add a new forest'' for a new domain or ``Add a domain controller to an existing domain''
		      \item Specify the root domain name (e.g., company.local)
		      \item Set the Forest and Domain functional levels (Windows Server 2016 or higher recommended)
		      \item Configure DNS and Global Catalog options (typically enabled by default)
		      \item Set Directory Services Restore Mode (DSRM) password
		      \item Review NetBIOS domain name
		      \item Specify paths for AD database, log files, and SYSVOL
		      \item Review settings and promote
		      \item Server will restart automatically
	      \end{itemize}

	\item \textbf{Post-Installation Configuration}
	      \begin{itemize}
		      \item Verify DNS is functioning correctly
		      \item Create Organizational Units (OUs) for logical organization
		      \item Configure Group Policy Objects (GPOs) as needed
		      \item Set up additional domain controllers for redundancy
		      \item Configure Active Directory Sites and Services if multi-site
		      \item Implement backup strategy for system state and Active Directory
		      \item Configure time synchronization (PDC Emulator should sync with external source)
		      \item Enable and configure Active Directory Recycle Bin for easier object recovery
	      \end{itemize}
\end{enumerate}

\subsection{Installation Steps - PowerShell}

\begin{enumerate}
	\item \textbf{Set Static IP Address}
	      \begin{lstlisting}[language=bash]
# View current network adapters
Get-NetAdapter

# Set static IP (adjust values for your environment)
New-NetIPAddress -InterfaceAlias "Ethernet" -IPAddress 10.0.1.10 -PrefixLength 24 -DefaultGateway 10.0.1.1

# Set DNS to localhost (127.0.0.1) and secondary
Set-DnsClientServerAddress -InterfaceAlias "Ethernet" -ServerAddresses 127.0.0.1,8.8.8.8
\end{lstlisting}

	\item \textbf{Rename Computer}
	      \begin{lstlisting}[language=bash]
# Rename the server
Rename-Computer -NewName "DC01" -Restart
\end{lstlisting}

	\item \textbf{Install AD DS Role}
	      \begin{lstlisting}[language=bash]
# Install Active Directory Domain Services role with management tools
Install-WindowsFeature -Name AD-Domain-Services -IncludeManagementTools

# Verify installation
Get-WindowsFeature | Where-Object {$_.Name -eq "AD-Domain-Services"}
\end{lstlisting}

	\item \textbf{Promote to Domain Controller (New Forest)}
	      \begin{lstlisting}[language=bash]
# Import the AD DS Deployment module
Import-Module ADDSDeployment

# Create new forest and promote to DC
Install-ADDSForest `
    -DomainName "company.local" `
    -DomainNetbiosName "COMPANY" `
    -ForestMode "WinThreshold" `
    -DomainMode "WinThreshold" `
    -InstallDns:$true `
    -CreateDnsDelegation:$false `
    -DatabasePath "C:\Windows\NTDS" `
    -LogPath "C:\Windows\NTDS" `
    -SysvolPath "C:\Windows\SYSVOL" `
    -NoRebootOnCompletion:$false `
    -Force:$true
\end{lstlisting}
	      \textit{Note: You'll be prompted for the SafeModeAdministratorPassword (DSRM password)}

	\item \textbf{Promote Additional Domain Controller (Existing Domain)}
	      \begin{lstlisting}[language=bash]
# Add DC to existing domain
Install-ADDSDomainController `
    -DomainName "company.local" `
    -InstallDns:$true `
    -Credential (Get-Credential "COMPANY\Administrator") `
    -DatabasePath "C:\Windows\NTDS" `
    -LogPath "C:\Windows\NTDS" `
    -SysvolPath "C:\Windows\SYSVOL" `
    -NoRebootOnCompletion:$false `
    -Force:$true
\end{lstlisting}

	\item \textbf{Post-Installation Verification}
	      \begin{lstlisting}[language=bash]
# Verify AD Web Services is running
Get-Service ADWS

# Check domain controller functionality
Get-ADDomainController

# Test AD replication (if multiple DCs)
repadmin /replsummary

# Verify DNS zones
Get-DnsServerZone

# Check SYSVOL replication
dfsrdiag replicationstate /all

# Verify FSMO roles
Get-ADDomain | Select-Object InfrastructureMaster, RIDMaster, PDCEmulator
Get-ADForest | Select-Object DomainNamingMaster, SchemaMaster
\end{lstlisting}

	\item \textbf{Create Organizational Units}
	      \begin{lstlisting}[language=bash]
# Create OUs for organization
New-ADOrganizationalUnit -Name "Users" -Path "DC=company,DC=local"
New-ADOrganizationalUnit -Name "Computers" -Path "DC=company,DC=local"
New-ADOrganizationalUnit -Name "Groups" -Path "DC=company,DC=local"
New-ADOrganizationalUnit -Name "Servers" -Path "DC=company,DC=local"
\end{lstlisting}

	\item \textbf{Configure Active Directory Recycle Bin}
	      \begin{lstlisting}[language=bash]
# Enable AD Recycle Bin (cannot be reversed)
Enable-ADOptionalFeature -Identity 'Recycle Bin Feature' `
    -Scope ForestOrConfigurationSet `
    -Target 'company.local' `
    -Confirm:$false
\end{lstlisting}

	\item \textbf{Configure Time Synchronization (PDC Emulator)}
	      \begin{lstlisting}[language=bash]
# Configure external time source on PDC Emulator
w32tm /config /manualpeerlist:"time.windows.com,0x8" /syncfromflags:manual /reliable:yes /update

# Restart Windows Time service
Restart-Service W32Time

# Force sync and check status
w32tm /resync
w32tm /query /status
\end{lstlisting}

	\item \textbf{Set Password Policy}
	      \begin{lstlisting}[language=bash]
# Configure default domain password policy
Set-ADDefaultDomainPasswordPolicy -Identity "company.local" `
    -ComplexityEnabled $true `
    -LockoutDuration "00:30:00" `
    -LockoutThreshold 5 `
    -MaxPasswordAge "90.00:00:00" `
    -MinPasswordAge "1.00:00:00" `
    -MinPasswordLength 12 `
    -PasswordHistoryCount 24
\end{lstlisting}
\end{enumerate}

\subsection{Security Best Practices}
\begin{itemize}
	\item Implement least privilege access for domain administrators
	\item Use separate administrative accounts for daily tasks vs. domain administration
	\item Enable and monitor security logs
	\item Regularly patch and update the domain controller
	\item Consider implementing tiered administrative model
	\item Use strong password policies and account lockout policies
	\item Utilize Advanced Threat Protection
\end{itemize}

\end{document}
