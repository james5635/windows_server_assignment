\documentclass[11pt, a4paper]{article}

% --- UNIVERSAL PREAMBLE BLOCK ---
\usepackage[a4paper, top=2.5cm, bottom=2.5cm, left=2cm, right=2cm]{geometry}
\usepackage{fontspec}
\usepackage[english, provide=*]{babel}

\babelprovide[import, onchar=ids fonts]{english}

% Set default/Latin font to Sans Serif in the main (rm) slot
\babelfont{rm}{Noto Sans}
\babelfont{sf}{Noto Sans}
\babelfont{tt}{Noto Sans Mono}


% Add because main language is not English (Standard Protocol) - though strictly English here, good for consistency
\usepackage{enumitem}
\setlist[itemize]{label=-}

% --- PACKAGES FOR THIS DOCUMENT ---
\usepackage{xcolor}
\usepackage{listings}
\usepackage{titlesec}
\usepackage{booktabs}
\usepackage{graphicx}
\usepackage{caption}

% Hyperref must be loaded last
\usepackage[hidelinks]{hyperref}
\hypersetup{
    colorlinks=true,
    linkcolor=blue,
    urlcolor=blue,
    pdftitle={Windows Server Deployment Guide on AWS Cloud}
}

% --- CODE LISTING CONFIGURATION ---
\definecolor{codegreen}{rgb}{0,0.6,0}
\definecolor{codegray}{rgb}{0.5,0.5,0.5}
\definecolor{codepurple}{rgb}{0.58,0,0.82}
\definecolor{backcolour}{rgb}{0.96,0.96,0.96}

\lstdefinestyle{mystyle}{
    backgroundcolor=\color{backcolour},
    commentstyle=\color{codegreen},
    keywordstyle=\color{magenta},
    numberstyle=\tiny\color{codegray},
    stringstyle=\color{codepurple},
    basicstyle=\ttfamily\footnotesize,
    breakatwhitespace=false,
    breaklines=true,
    captionpos=b,
    keepspaces=true,
    numbers=left,
    numbersep=5pt,
    showspaces=false,
    showstringspaces=false,
    showtabs=false,
    tabsize=2,
    frame=single
}

\lstset{style=mystyle}

% --- DOCUMENT START ---
\title{\textbf{Windows Server Deployment Guide on AWS Cloud}}
% \author{Sou Chanrojame}
\author{Sou Chanrojame, Orn Pheakdey, Long Neron, Then Sivthean, Le Sreyma}
\date{\today}

\begin{document}

\maketitle

\begin{abstract}
	This document provides a comprehensive step-by-step guide for deploying various Windows Server roles and services on the AWS Cloud infrastructure, including configuration details for EC2 instances, security groups, and storage optimization.
\end{abstract}


\tableofcontents
\newpage

\phantomsection
\addcontentsline{toc}{section}{Prerequisites}
\section*{Prerequisites}

\subsection*{AWS Account Setup}
\begin{itemize}
	\item Active AWS account with appropriate permissions
	\item VPC configured with public and private subnets
	\item Security groups properly configured
	\item Key pairs created for RDP access
	\item IAM roles for EC2 instances
\end{itemize}

\subsection*{General Windows Server Launch Steps}
\begin{enumerate}
	\item Navigate to EC2 Dashboard in AWS Console
	\item Click ``Launch Instance''
	\item Select Windows Server AMI (2019/2022 recommended)
	\item Choose instance type based on workload
	\item Configure instance details (VPC, subnet, IAM role)
	\item Add storage as needed
	\item Configure security groups
	\item Review and launch with key pair
\end{enumerate}

% \hrule
\newpage

\section{File Server}

\subsection{AWS Configuration}
Instance Name: File Server\\
Instance Type: t3.medium or larger\\
Storage: EBS volumes with provisioned IOPS for performance\\
Security Group Ports: tcp 445 (SMB), tcp 3389 (RDP)

\subsection{Implementation}

\begin{lstlisting}[language=bash]
Install-WindowsFeature -Name FS-FileServer
New-Item -Path "C:\FileShare" -ItemType Directory -Force
New-SmbShare -Name "PublicShare" -Path "C:\FileShare" -FullAccess "Everyone"
echo hello > C:\FileShare\hello.txt
Write-Host "File Server setup complete."
\end{lstlisting}


\subsection{Usage}
\begin{itemize}
	\item connect to Proxy Server or other windows server
	\item open file explorer and \verb|\\<public dns of File Server>|
	\item enter credential
\end{itemize}

% \includegraphics[width=\textwidth]{static/file_server/demo.png}

\begin{figure}[htbp]
	\centering
	\includegraphics[width=\textwidth]{static/file_server/demo.png}
	\caption{Successfully connected to file server}
	\label{fig:file-server}
\end{figure}

\newpage

\section{Proxy Server (Caching, Control Access)}

\subsection{AWS Configuration}
Instance Name: Proxy Server\\
Instance Type: t3.medium\\
Security Group Ports: tcp 3128, tcp 3389 (RDP)

\subsection{Implementation}
\begin{lstlisting}[language=bash]
Set-ExecutionPolicy Bypass -Scope Process -Force; [System.Net.ServicePointManager]::SecurityProtocol = [System.Net.ServicePointManager]::SecurityProtocol -bor 3072; iex ((New-Object System.Net.WebClient).DownloadString('https://community.chocolatey.org/install.ps1'))
choco install squid -y
Invoke-WebRequest https://raw.githubusercontent.com/james5635/windows_server_assignment/refs/heads/main/config/squid.conf -OutFile "C:\Squid\etc\squid\squid.conf"
Restart-Service squidsrv
Write-Host "Proxy Server setup complete."
\end{lstlisting}


\subsection{Usage}
\begin{itemize}
	\item connect to File Server or other windows server
	\item change proxy address to \verb|<private ip of Proxy Server>| with port 3128
	\item open browser
	\item visit youtube.com => allow
	\item visit facebook.com => blocked
\end{itemize}


\begin{figure}[htbp]
	\includegraphics[width=\textwidth]{static/proxy_server/allow_youtube.png}
	\caption{Allow access to youtube}

\end{figure}
\clearpage

\begin{figure}[htbp]
	\includegraphics[width=\textwidth]{static/proxy_server/block_facebook.png}
	\caption{Block access to facebook}
\end{figure}

\clearpage

\section{DNS Server}

\subsection{AWS Configuration}
Instance Name: DNS Server\\
Instance Type: t3.medium\\
Security Group Ports: tcp/udp 53, tcp 3389 (RDP)

\subsection{Implementation}
\begin{lstlisting}[language=bash]
Install-WindowsFeature DNS -IncludeManagementTools

Add-DnsServerPrimaryZone -Name "example.internal" -ZoneFile "example.internal.dns"
Add-DnsServerPrimaryZone -Name "devspeed.com" -ZoneFile "devspeed.com.dns"

Add-DnsServerResourceRecordA -Name 'server1' -ZoneName 'example.internal' -IPv4Address '10.0.1.10'
Add-DnsServerResourceRecordA -Name 'console' -ZoneName 'devspeed.com' -IPv4Address '192.168.1.11'
Add-DnsServerResourceRecordA -Name 'go' -ZoneName 'devspeed.com' -IPv4Address '192.168.1.29'
Add-DnsServerResourceRecordA -Name 'blog' -ZoneName 'devspeed.com' -IPv4Address '192.168.1.30'
Add-DnsServerResourceRecordA -Name 'shop' -ZoneName 'devspeed.com' -IPv4Address '192.168.1.31'
Add-DnsServerResourceRecordA -Name 'support' -ZoneName 'devspeed.com' -IPv4Address '192.168.1.32'
Add-DnsServerResourceRecordA -Name 'mail' -ZoneName 'devspeed.com' -IPv4Address '192.168.1.33'
Add-DnsServerResourceRecordA -Name 'www' -ZoneName 'devspeed.com' -IPv4Address '192.168.1.34'
Add-DnsServerResourceRecordA -Name 'www2' -ZoneName 'devspeed.com' -IPv4Address '192.168.1.100'

# For example, forwarding to Google DNS:
# Add-DnsServerForwarder -IPAddress "8.8.8.8" -PassThru
# Or forwarding to AWS VPC DNS (recommended for EC2). This is the Amazon-provided DNS resolver for VPCs.
# Add-DnsServerForwarder -IPAddress "169.254.169.253" -PassThru

New-NetFirewallRule -DisplayName "Allow DNS TCP 53" -Direction Inbound -Protocol TCP -LocalPort 53 -Action Allow
New-NetFirewallRule -DisplayName "Allow DNS UDP 53" -Direction Inbound -Protocol UDP -LocalPort 53 -Action Allow

\end{lstlisting}



\subsection{Usage}
\begin{itemize}
	\item open Terminal
	\item dog www.devspeed.com `@<public ip of dns server>`
	\item dog go.devspeed.com `@<public ip of dns server>`
	\item nslookup shop.devspeed.com `<public ip of dns server>`
	\item nslookup mail.devspeed.com `<public ip of dns server>`
\end{itemize}
\clearpage
\begin{figure}[htbp]
	\includegraphics[width=\textwidth]{static/dns_server/connect_dns_server.png}
	\caption{Dns server response with respective IP address}
\end{figure}

\clearpage

\section{DHCP Server}

\subsection{AWS Configuration}
\textbf{Instance Type:} t3.small\\
\textbf{Note:} AWS VPC provides DHCP by default; custom DHCP server is optional.

\subsection{Implementation Steps}

\begin{enumerate}
	\item \textbf{Launch Windows Server Instance}

	\item \textbf{Install DHCP Server Role}
	      \begin{lstlisting}[language=bash]
Install-WindowsFeature -Name DHCP -IncludeManagementTools
Add-DhcpServerInDC -DnsName "dhcp.yourdomain.local"
\end{lstlisting}

	\item \textbf{Configure DHCP Scope}
	      \begin{lstlisting}[language=bash]
Add-DhcpServerv4Scope -Name "Internal Network" -StartRange 10.0.1.100 -EndRange 10.0.1.200 -SubnetMask 255.255.255.0

Set-DhcpServerv4OptionValue -ScopeId 10.0.1.0 -Router 10.0.1.1
Set-DhcpServerv4OptionValue -ScopeId 10.0.1.0 -DnsServer 10.0.1.10
\end{lstlisting}

	\item \textbf{Configure Reservations}
	      \begin{lstlisting}[language=bash]
Add-DhcpServerv4Reservation -ScopeId 10.0.1.0 -IPAddress 10.0.1.50 -ClientId "00-11-22-33-44-55" -Description "Print Server"
\end{lstlisting}

	\item \textbf{Authorize DHCP Server}
	      \begin{lstlisting}[language=bash]
Add-DhcpServerInDC -DnsName "dhcp.yourdomain.local" -IPAddress 10.0.1.10
\end{lstlisting}
\end{enumerate}

\subsection{Best Practices}
\begin{itemize}
	\item Consider using AWS-provided DHCP for simplicity
	\item Deploy DHCP failover for redundancy
	\item Use DHCP policies for different device types
	\item Monitor DHCP lease utilization
\end{itemize}

\newpage

\section{VPN Server}

\subsection{AWS Configuration}
Instance Name: VPN Server\\
Instance Type: t3.small to t3.medium\\
Security Group Ports: udp 1194, tcp 3389 (RDP)\\
Elastic IP: Required for consistent endpoint

\subsection{Implementation}

\begin{lstlisting}[language=bash]

Set-ExecutionPolicy Bypass -Scope Process -Force; [System.Net.ServicePointManager]::SecurityProtocol = [System.Net.ServicePointManager]::SecurityProtocol -bor 3072; iex ((New-Object System.Net.WebClient).DownloadString('https://community.chocolatey.org/install.ps1'))
choco install openvpn -y  --package-parameters=" /AddToDesktop /Gui /GuiOnLogon /EasyRsa /DcoDriver /TapDriver /WintunDriver /OpenSSL /Service "
choco install caddy 7zip -y

$cfg = "C:\Program Files\OpenVPN\config"

Invoke-WebRequest https://raw.githubusercontent.com/james5635/windows_server_assignment/refs/heads/main/config/openvpn/server/dh.pem -OutFile "$cfg\dh.pem"
Invoke-WebRequest https://raw.githubusercontent.com/james5635/windows_server_assignment/refs/heads/main/config/openvpn/server/server.crt -OutFile "$cfg\server.crt"
Invoke-WebRequest https://raw.githubusercontent.com/james5635/windows_server_assignment/refs/heads/main/config/openvpn/server/server.key -OutFile "$cfg\server.key"
Invoke-WebRequest https://raw.githubusercontent.com/james5635/windows_server_assignment/refs/heads/main/config/openvpn/server/ca.crt -OutFile "$cfg\ca.crt"
Invoke-WebRequest https://raw.githubusercontent.com/james5635/windows_server_assignment/refs/heads/main/config/openvpn/server/server.ovpn -OutFile "$cfg\server.ovpn"

Set-Service -Name OpenVPNService -StartupType Automatic
Start-Service -Name OpenVPNService

New-NetFirewallRule -DisplayName "OpenVPN" -Direction Inbound -Protocol UDP -LocalPort 1194 -Action Allow
New-NetFirewallRule -DisplayName "ShareFileOpenVPN" -Direction Inbound -Protocol TCP -LocalPort 80 -Action Allow

# don't know why it doesn't working. (The server doesn't run)
# sleep -Seconds 5 # still not working
# Start-Process "C:\Program Files\OpenVPN\bin\openvpn-gui.exe"

# # for client to openvpn server
# $cfg = "C:\Program Files\OpenVPN\config"
# Invoke-WebRequest <url> "$cfg\client1.ovpn"
# Invoke-WebRequest <url> "$cfg\ca.crt"
# Invoke-WebRequest <url> "$cfg\client1.crt"
# Invoke-WebRequest <url> "$cfg\client1.key"

\end{lstlisting}

\subsection{Usage}
\begin{itemize}
	\item connect to VPN Server
	\item Open \verb|OpenVPN GUI| to start the server
	\item connect to File Server
	\item change \verb|YOUR_PUBLIC_IP| in \verb|C:\Program Files\OpenVPN\config\client1.ovpn| \\
	      to public ip of the VPN Server
	\item Open \verb|OpenVPN GUI| to connect to the server
	\item Open powershell and type \verb|ipconfig| and will see something like:

	      \begin{lstlisting}{language=bash}
Unknown adapter OpenVPN Data Channel Offload:

   Connection-specific DNS Suffix  . :
   Link-local IPv6 Address . . . . . : fe80::f729:5f67:58f2:7253%17
   IPv4 Address. . . . . . . . . . . : 10.8.0.6
   Subnet Mask . . . . . . . . . . . : 255.255.255.252
   Default Gateway . . . . . . . . . :
\end{lstlisting}

\end{itemize}

\begin{figure}[htbp]
	\includegraphics[width=\textwidth]{static/vpn_server/connect_vpn_server.png}
	\caption{vpn client successfully connect to VPN server}
\end{figure}

\newpage

\section{Terminal Server (Thin Clients)}

\subsection{AWS Configuration}
Instance Name: Terminal Server\\
Instance Type: t3.medium\\
Security Group Ports: tcp 3389 (RDP)

\subsection{Implementation}

\begin{lstlisting}[language=bash]
# Install Remote Desktop Session Host Role
Install-WindowsFeature RDS-RD-Server
# Enable Multiple Sessions
Set-ItemProperty -Path "HKLM:\SYSTEM\CurrentControlSet\Control\Terminal Server" -Name fSingleSessionPerUser -Value 0
# Allow RDP Firewall Rule
Enable-NetFirewallRule -DisplayGroup "Remote Desktop"
Write-Host "Terminal Server (RDS Session Host) setup complete."
\end{lstlisting}

\subsection{Usage}
\begin{itemize}
	\item connect to the server with RDP
\end{itemize}

\begin{figure}[htbp]
	\includegraphics[width=\textwidth]{static/terminal_server/connect_terminal_server.png}
	\caption{Successfully connect to terminal server}
\end{figure}

\newpage

\section{Web Server}

\subsection{AWS Configuration}
Instance Name: Web Server\\
Instance Type: t3.medium\\
Security Group Ports: tcp 80 (HTTP), tcp 443 (HTTPS), tcp 3389 (RDP)\\

\subsection{Implementation}

\begin{lstlisting}{language=bash}
Install-WindowsFeature -Name Web-Server -IncludeManagementTools
# Install-WindowsFeature -Name Web-Asp-Net45, Web-Net-Ext45
# Install-WindowsFeature -Name Web-Mgmt-Console
Invoke-WebRequest https://raw.githubusercontent.com/james5635/windows_server_assignment/refs/heads/main/static/devspeed.html -OutFile "C:\inetpub\wwwroot\index.html"
\end{lstlisting}

\subsection{Usage}
\begin{itemize}
	\item visit public dns of the server
\end{itemize}

\begin{figure}[htbp]
	\includegraphics[width=\textwidth]{static/web_server/connect_web_server.png}
	\caption{visiting the web page}
\end{figure}

\newpage

\section{Mail Server}

\subsection{AWS Configuration}
\textbf{Instance Type:} t3.medium\\
\textbf{Security Group Ports:} 25 (SMTP), 110 (POP3), 143 (IMAP), 587 (Submission), 993 (IMAPS), 995 (POP3S)\\
\textbf{Elastic IP:} Required\\
\textbf{Note:} AWS blocks port 25 by default; request removal.

\subsection{Implementation}

\begin{lstlisting}[language=bash]
# mail server
# - apache james (smpt server [port: 25], imap server [port: 143])
# - swaks (smpt client)
# - thunderbird (email client)

Set-ExecutionPolicy Bypass -Scope Process -Force; [System.Net.ServicePointManager]::SecurityProtocol = [System.Net.ServicePointManager]::SecurityProtocol -bor 3072; iex ((New-Object System.Net.WebClient).DownloadString('https://community.chocolatey.org/install.ps1'))
choco install openjdk strawberryperl thunderbird notepadplusplus telnet -y
Invoke-WebRequest https://www.jetmore.org/john/code/swaks/files/swaks-20240103.0/swaks -OutFile swaks.pl
Invoke-WebRequest https://dlcdn.apache.org/james/server/3.9.0/james-server-spring-app-3.9.0-app.zip -OutFile james-server-spring-app-3.9.0-app.zip
Expand-Archive james-server-spring-app-3.9.0-app.zip -DestinationPath james-server-spring-app-3.9.0-app

Invoke-WebRequest https://raw.githubusercontent.com/james5635/windows_server_assignment/refs/heads/main/config/smtpserver.xml -OutFile "james-server-spring-app-3.9.0-app\james-server-spring-app-3.9.0\conf\smtpserver.xml"
Invoke-WebRequest https://raw.githubusercontent.com/james5635/windows_server_assignment/refs/heads/main/config/imapserver.xml -OutFile "james-server-spring-app-3.9.0-app\james-server-spring-app-3.9.0\conf\imapserver.xml"

Import-Module $env:ChocolateyInstall\helpers\chocolateyProfile.psm1
refreshenv

# cd "james-server-spring-app-3.9.0-app\james-server-spring-app-3.9.0\bin\"

$james = "C:\Windows\System32\james-server-spring-app-3.9.0-app\james-server-spring-app-3.9.0\bin\james.bat"
$james_cli = "C:\Windows\System32\james-server-spring-app-3.9.0-app\james-server-spring-app-3.9.0\bin\james-cli.ps1"

Invoke-WebRequest https://raw.githubusercontent.com/james5635/windows_server_assignment/refs/heads/main/scripts/james-cli.ps1 -OutFile $james_cli

function Wait-JamesReady {
    param(
        [int]$Port = 9999,
        [string]$Url = "localhost",
        [int]$TimeoutSeconds = 120
    )

    $start = Get-Date

    Write-Host "Waiting for James server to open JMX on port $Port..."

    while ((Get-Date) -lt $start.AddSeconds($TimeoutSeconds)) {
        if ((Test-NetConnection -ComputerName $Url -Port $Port -WarningAction SilentlyContinue).TcpTestSucceeded) {
            Write-Host "James server is ready!"
            return
        }

        Start-Sleep -Seconds 2
    }

    throw "Timeout: James server did not open JMX port $Port"
}

& $james install
& $james start

# need to wait james server completely ready
Wait-JamesReady

& $james_cli -username james-admin -password changeme adddomain example.local
& $james_cli -username james-admin -password changeme adduser mike@example.local mike
& $james_cli -username james-admin -password changeme adduser joe@example.local joe
& $james_cli -username james-admin -password changeme adduser leng@example.local leng
& $james_cli -username james-admin -password changeme adduser jame@example.local jame

New-NetFirewallRule -DisplayName "Allow smtp 25" -Direction Inbound -Protocol TCP -LocalPort 25 -Action Allow
New-NetFirewallRule -DisplayName "Allow imap 143" -Direction Inbound -Protocol TCP -LocalPort 143 -Action Allow

\end{lstlisting}

\subsection{Usage}
\begin{itemize}
	\item Open Thunderbird
	\item Username: \texttt{jame@example.local}
	\item Password: \texttt{jame}
	\item Username: \texttt{mike@example.local}
	\item Password: \texttt{mike}

	\item IMAP
	      \begin{itemize}
		      \item Hostname: \texttt{<public-ip-of-mail-server>}
		      \item Port: 143
	      \end{itemize}

	\item SMTP
	      \begin{itemize}
		      \item Hostname: \texttt{<public-ip-of-mail-server>}
		      \item Port: 25
	      \end{itemize}
\end{itemize}

\clearpage
\begin{figure}[htbp]
	\includegraphics[width=\textwidth]{static/mail_server/connect_mail_server.png}
	\caption{Two users sending messages to each other}
\end{figure}

\clearpage

\section{Database Server}

\subsection{AWS Configuration}
Instance Name: Database Server\\
Instance Type: t3.medium or larger (memory-optimized)\\
Storage: EBS with provisioned IOPS or io2\\
Security Group Ports:
\begin{itemize}
	\item PostgreSQL: tcp 5432
	\item SQL Server: tcp 1433
	\item MongoDB: tcp 27017
	\item RDP: tcp 3389
\end{itemize}

\subsection{Implementation}

\subsubsection{PostgreSQL}

\begin{lstlisting}[language=bash]
 # postgresql
 Set-ExecutionPolicy Bypass -Scope Process -Force; [System.Net.ServicePointManager]::SecurityProtocol = [System.Net.ServicePointManager]::SecurityProtocol -bor 3072; iex ((New-Object System.Net.WebClient).DownloadString('https://community.chocolatey.org/install.ps1'))
 # username postgres
 choco install postgresql11 --params '/Password:test /AllowRemote' -y
 Restart-Service postgresql-x64-11
 New-NetFirewallRule -DisplayName "Allow PostgreSQL 5432" `
   -Direction Inbound `
   -Protocol TCP `
   -LocalPort 5432 `
   -Action Allow
\end{lstlisting}

\subsubsection{SQL Server}
\begin{lstlisting}[language=bash]
 # sql server
 Set-ExecutionPolicy Bypass -Scope Process -Force; [System.Net.ServicePointManager]::SecurityProtocol = [System.Net.ServicePointManager]::SecurityProtocol -bor 3072; iex ((New-Object System.Net.WebClient).DownloadString('https://community.chocolatey.org/install.ps1'))
 choco install mssqlserver2014express -y
 choco install sqlserver-cmdlineutils -y

 # Get access to SqlWmiManagement DLL on the machine with SQL
 # we are on, which is where SQL Server was installed.
 # Note: This is installed in the GAC by SQL Server Setup.

 # Load the WMI assembly
 [System.Reflection.Assembly]::LoadWithPartialName('Microsoft.SqlServer.SqlWmiManagement') | Out-Null

 # Connect to local SQL Server machine
 $wmi = New-Object 'Microsoft.SqlServer.Management.Smo.Wmi.ManagedComputer' 'localhost'

 # Get TCP protocol for SQLEXPRESS
 $tcp = $wmi.ServerInstances['SQLEXPRESS'].ServerProtocols['Tcp']
 $tcp.IsEnabled = $true

 # Disable dynamic ports on each IP entry
 foreach ($ip in $tcp.IPAddresses) {
     $ip.IPAddressProperties["TcpDynamicPorts"].Value = ""
 }

 # Set static port 1433 on ALL IPs (including IPAll)
 foreach ($ip in $tcp.IPAddresses) {
     $ip.IPAddressProperties["TcpPort"].Value = "1433"
 }

 # Apply configuration
 $tcp.Alter()

 # You need to restart SQL Server for the change to persist
 # -Force takes care of any dependent services, like SQL Agent.
 # Note: If the instance is named, replace MSSQLSERVER with MSSQL$ followed by
 # the name of the instance (e.g., MSSQL$MYINSTANCE)

 Restart-Service -Name 'MSSQL$SQLEXPRESS' -Force

 & "C:\Program Files\Microsoft SQL Server\Client SDK\ODBC\110\Tools\Binn\SQLCMD.EXE"  -S .\SQLEXPRESS -Q "ALTER LOGIN sa ENABLE; ALTER LOGIN sa WITH PASSWORD = 'mypassword@2025';"
 Set-ItemProperty -Path "HKLM:\SOFTWARE\Microsoft\Microsoft SQL Server\MSSQL12.SQLEXPRESS\MSSQLServer" -Name LoginMode -Value 2
 Restart-Service -Name 'MSSQL$SQLEXPRESS' -Force

 New-NetFirewallRule -DisplayName "Allow SQL Server 1433" `
   -Direction Inbound `
   -Protocol TCP `
   -LocalPort 1433 `
   -Action Allow

\end{lstlisting}
\subsubsection{MongoDB}
\begin{lstlisting}[language=bash]
 # mongodb
 Set-ExecutionPolicy Bypass -Scope Process -Force; [System.Net.ServicePointManager]::SecurityProtocol = [System.Net.ServicePointManager]::SecurityProtocol -bor 3072; iex ((New-Object System.Net.WebClient).DownloadString('https://community.chocolatey.org/install.ps1'))
 choco install mongodb -y
 choco install mongodb-shell -y
 Invoke-WebRequest https://raw.githubusercontent.com/james5635/windows_server_assignment/refs/heads/main/config/mongod.cfg -OutFile "C:\Program Files\MongoDB\Server\8.2\bin\mongod.cfg"
 Restart-Service MongoDB
 New-NetFirewallRule -DisplayName "Allow MongoDB 27017" `
   -Direction Inbound `
   -Protocol TCP `
   -LocalPort 27017 `
   -Action Allow
\end{lstlisting}

\subsection{Usage}
\begin{itemize}
	\item MongoDB
	      \begin{itemize}
		      \item mongosh \texttt{<public ip of mongodb server>}
	      \end{itemize}
	\item SQL Server
	      \begin{itemize}
		      \item /opt/mssql-tools18/bin/sqlcmd -S \texttt{<public ip of sql server>} -C -U sa -P mypassword@2025
	      \end{itemize}
	\item PostgreSQL
	      \begin{itemize}
		      \item psql -h \texttt{<public ip of postgresql server>} -U postgres
		      \item password: test
	      \end{itemize}
\end{itemize}

\begin{figure}[htbp]
	\includegraphics[width=\textwidth]{static/database_server/postgresql.png}
	\caption{Client using postgresql server}
\end{figure}

\begin{figure}[htbp]
	\includegraphics[width=\textwidth]{static/database_server/sql_server.png}
	\caption{client using sql server}
\end{figure}
\clearpage
\begin{figure}[htbp]
	\includegraphics[width=\textwidth]{static/database_server/mongodb.png}
	\caption{client using mongodb server}
\end{figure}

\clearpage

\section{Backup Server}

\subsection{AWS Configuration}
Instance Type: t3.medium\\
Security Group Ports: tcp 3389 (RDP)\\
Storage: Large EBS volumes or S3 integration\\
IAM Role: Permissions for S3, EBS snapshots

\subsection{Implementation}
\begin{lstlisting}[language=bash]
Set-ExecutionPolicy Bypass -Scope Process -Force; [System.Net.ServicePointManager]::SecurityProtocol = [System.Net.ServicePointManager]::SecurityProtocol -bor 3072; iex ((New-Object System.Net.WebClient).DownloadString('https://community.chocolatey.org/install.ps1'))
choco install awscli -y

Import-Module $env:ChocolateyInstall\helpers\chocolateyProfile.psm1
refreshenv

$bucket = (New-Guid).ToString()
echo $bucket > "C:\bucket.txt"

aws s3 mb "s3://$bucket"

Invoke-WebRequest https://raw.githubusercontent.com/james5635/windows_server_assignment/refs/heads/main/scripts/S3Backup.ps1 -OutFile "C:\S3Backup.ps1"
Invoke-WebRequest https://raw.githubusercontent.com/james5635/windows_server_assignment/refs/heads/main/scripts/S3BackupSchedule.ps1 -OutFile "C:\S3BackupSchedule.ps1"
Invoke-WebRequest https://raw.githubusercontent.com/james5635/windows_server_assignment/refs/heads/main/scripts/S3BackupScheduleNightly.ps1 -OutFile "C:\S3BackupScheduleNightly.ps1"

& "C:\S3BackupSchedule.ps1"
& "C:\S3BackupScheduleNightly.ps1"

\end{lstlisting}

\subsection{Usage}
\begin{itemize}
	\item automatically backup \verb|"C:\inetpub"| to s3 bucket every 1 minute and every night at 2 a.m.
	\item check out the s3 bucket
\end{itemize}
\clearpage
\begin{figure}[htbp]
	\includegraphics[width=\textwidth]{static/backup_server/backup_server.png}
	\caption{automatically backup to s3 bucket}
\end{figure}

\clearpage

\section{Load Balancing}

\subsection{AWS Configuration}
Service: Application Load Balancer (ALB)\\
Target Group: Multiple Windows Server instances\\
Instance Type: t3.medium\\
Security Group Ports: tcp 3389 (RDP)

\subsection{Implementation Steps}
\subsubsection{Server 1}
\begin{lstlisting}[language=bash]
Start-Transcript -Path "C:\WebServer.log" -Append
Install-WindowsFeature -Name Web-Server
# echo "Server 1" > C:\inetpub\wwwroot\index.html
\end{lstlisting}
\subsubsection{Server 2}
\begin{lstlisting}[language=bash]
Install-WindowsFeature -Name Web-Server -IncludeManagementTools
# Install-WindowsFeature -Name Web-Asp-Net45, Web-Net-Ext45
# Install-WindowsFeature -Name Web-Mgmt-Console
Invoke-WebRequest https://raw.githubusercontent.com/james5635/windows_server_assignment/refs/heads/main/static/devspeed.html -OutFile "C:\inetpub\wwwroot\index.html"
\end{lstlisting}

\subsection{Usage}
\begin{itemize}
\item manually configure load balancer (application load balancer)
\item name: devspeedlb
\item select all availability Zones and subnets
\item create target group (name: web)
\item
    \begin{itemize}
        \item include WebServerMailServer \& Docker
    \end{itemize}
\item choose 'web' as target group
\item add tcp (port 80) to inbound rule of the load balancer's security group
\end{itemize}
\clearpage
\begin{figure}[htbp]
	\includegraphics[width=\textwidth]{static/load_balancing/server_1.png}
	\caption{load balance to server 1}
\end{figure}
\begin{figure}[htbp]
	\includegraphics[width=\textwidth]{static/load_balancing/server_2.png}
	\caption{load balance to server 2}
\end{figure}
\newpage

\section{Failover Cluster}

\subsection{AWS Configuration}
\textbf{Instance Type:} r5.xlarge or larger\\
\textbf{Storage:} Shared storage using FSx for Windows or S3\\
\textbf{Network:} Placement groups for low latency\\
\textbf{Security Group:} Allow cluster communication ports

\subsection{Implementation Steps}

\begin{enumerate}
	\item \textbf{Launch Multiple Windows Server Instances}
	      \begin{itemize}
		      \item Deploy in same VPC, different availability zones
		      \item Use placement group for low latency
	      \end{itemize}

	\item \textbf{Install Failover Clustering Feature}
	      \begin{lstlisting}[language=bash]
Install-WindowsFeature -Name Failover-Clustering -IncludeManagementTools
\end{lstlisting}

	\item \textbf{Configure Shared Storage}
	      \begin{itemize}
		      \item \textbf{Option A: Amazon FSx for Windows File Server}\\
		            Create FSx file system; Mount on all cluster nodes.
		      \item \textbf{Option B: EBS Multi-Attach (io2 only)}\\
		            Attach same EBS volume to multiple instances; Initialize as cluster shared volume.
	      \end{itemize}

	\item \textbf{Create Failover Cluster}
	      \begin{lstlisting}[language=bash]
# Validate cluster configuration
Test-Cluster -Node "Node1", "Node2"

# Create cluster
New-Cluster -Name "MyCluster" -Node "Node1", "Node2" -StaticAddress "10.0.1.100" -NoStorage
\end{lstlisting}

	\item \textbf{Configure Cluster Quorum}
	      \begin{lstlisting}[language=bash]
Set-ClusterQuorum -NodeAndFileShareMajority "\\FSx\Witness"
\end{lstlisting}

	\item \textbf{Add Clustered Role}
	      \begin{lstlisting}[language=bash]
# For SQL Server
Add-ClusterServerRole -Name "SQL-Cluster" -Storage "Cluster Disk 1"
\end{lstlisting}

	\item \textbf{Configure Secondary Private IP}
	      \begin{itemize}
		      \item Assign secondary private IP to ENI
		      \item Configure in cluster as virtual IP
	      \end{itemize}
\end{enumerate}

\subsection{Common Cluster Types in AWS}
\begin{description}
	\item[SQL Server Failover Cluster] Use FSx for shared storage; Configure SQL Server on cluster nodes; Set up availability group for database replication.
	\item[File Server Cluster] Use FSx or S3 for storage; Configure highly available file shares.
\end{description}

\subsection{Best Practices}
\begin{itemize}
	\item Use Amazon FSx for Windows File Server for shared storage
	\item Deploy cluster nodes in different availability zones
	\item Use Elastic IP or Network Load Balancer for client access
	\item Monitor cluster health with CloudWatch
	\item Regular testing of failover scenarios
	\item Consider Amazon RDS Multi-AZ for database clustering
\end{itemize}

\newpage

\section{FTP Server}

\subsection{AWS Configuration}
Instance Name: FTP Server\\
Instance Type: t3.small to t3.medium\\
Security Group Ports: tcp 21, tcp 50000-51000, tcp 3389 (RDP)\\

\subsection{Implementation}

\begin{lstlisting}[language=bash]
#Install IIS Feature
Install-WindowsFeature -Name Web-Server -IncludeManagementTools

#Install FTP feature
Install-WindowsFeature -Name Web-Ftp-Server -IncludeAllSubFeature -IncludeManagementTools -Verbose

#Creating new FTP site
$SiteName = "Demo FTP Site"
$RootFolderpath = "C:\inetpub\ftproot"
$PortNumber = 21
$FTPUserGroupName = "Demo FTP Users Group"
$FTPUserName = "jame"
$FTPPassword = ConvertTo-SecureString "Mypassword@2025" -AsPlainText -Force

if (!(Test-Path $RootFolderpath)) {
    # if the folder doesn't exist
    New-Item -Path $RootFolderpath -ItemType Directory # create the folder
}

New-WebFtpSite -Name $SiteName -PhysicalPath $RootFolderpath -Port $PortNumber -Verbose -Force

#Creating the local Windows group
if (!(Get-LocalGroup $FTPUserGroupName  -ErrorAction SilentlyContinue)) {
    #if the group doesn't exist
    New-LocalGroup -Name $FTPUserGroupName `
        -Description "Members of this group can connect to FTP server" #create the group
}

# Creating an FTP user
If (!(Get-LocalUser $FTPUserName -ErrorAction SilentlyContinue)) {
    New-LocalUser -Name $FTPUserName -Password $FTPPassword `
        -Description "User account to access FTP server" `
        -UserMayNotChangePassword
}

# Add the created FTP user to the group Demo FTP Users Group
Add-LocalGroupMember -Name $FTPUserGroupName -Member $FTPUserName -ErrorAction SilentlyContinue

# Enabling basic authentication on the FTP site
$param = @{
    Path    = 'IIS:\Sites\Demo FTP Site'
    Name    = 'ftpserver.security.authentication.basicauthentication.enabled'
    Value   = $true
    Verbose = $True
}
Set-ItemProperty @param

# Adding authorization rule to allow FTP users
# in the FTP group to access the FTP site
$param = @{
    PSPath   = 'IIS:\'
    Location = $SiteName
    Filter   = '/system.ftpserver/security/authorization'
#    Value    = @{ accesstype = 'Allow'; roles = $FTPUserGroupName; permissions = 1 }
    Value    = @{ accesstype = 'Allow'; roles = $FTPUserGroupName; permissions = 3 }
}

Add-WebConfiguration @param

# Changing SSL policy of the FTP site
'ftpServer.security.ssl.controlChannelPolicy', 'ftpServer.security.ssl.dataChannelPolicy' |
ForEach-Object {
    Set-ItemProperty -Path "IIS:\Sites\Demo FTP Site" -Name $_ -Value $false
}

# can be 'ReadAndExecute', 'Modify'
$ACLObject = Get-Acl -Path $RootFolderpath
$ACLObject.SetAccessRule(
    ( # Access rule object
        New-Object System.Security.AccessControl.FileSystemAccessRule(
            $FTPUserGroupName,
            'FullControl',
            'ContainerInherit,ObjectInherit',
            'None',
            'Allow'
        )
    )
)
Set-Acl -Path $RootFolderpath -AclObject $ACLObject

# Checking the NTFS permissions on the FTP root folder
Get-Acl -Path $RootFolderpath | ForEach-Object Access

# Test FTP Port and FTP access
Test-NetConnection -ComputerName localhost -Port 21


Set-WebConfigurationProperty -pspath 'MACHINE/WEBROOT/APPHOST'  -filter "system.ftpServer/firewallSupport" -name "lowDataChannelPort" -value 50000
Set-WebConfigurationProperty -pspath 'MACHINE/WEBROOT/APPHOST'  -filter "system.ftpServer/firewallSupport" -name "highDataChannelPort" -value 51000

Set-WebConfigurationProperty -pspath 'MACHINE/WEBROOT/APPHOST' -filter "system.applicationHost/sites/site[@name='Demo FTP Site']/ftpServer/firewallSupport" -name "externalIp4Address" -value "${MyEIP}"

# Control port
New-NetFirewallRule -DisplayName "Allow FTP 21" -Direction Inbound -Protocol TCP -LocalPort 21 -Action Allow

# Passive ports
New-NetFirewallRule -DisplayName "Allow FTP Passive Ports" -Direction Inbound -Protocol TCP -LocalPort 50000-51000 -Action Allow

Restart-Service FTPSVC

# C:\Windows\System32\config\systemprofile\AppData\Local\Temp\EC2Launch380802110\UserScript.ps1
# apple

echo hello > "C:\inetpub\ftproot\hello.txt"

\end{lstlisting}

\subsection{Usage}
\begin{itemize}
\item use filezilla (port 21)
\item username: jame
\item password: Mypassword@2025
\end{itemize}

\begin{figure}[htbp]
	\includegraphics[width=\textwidth]{static/ftp_server/ftp_server.png}
	\caption{FileZilla successfully connected to ftp server}
\end{figure}
\clearpage

\section{Container (Docker)}

\subsection{AWS Configuration}
Instance Type: t3.medium or larger\\
Operating System: Windows Server 2019/2022 with Containers\\
Security Group Ports: Custom ports based on containerized applications

\subsection{Implementation}

\begin{lstlisting}[language=bash]
Invoke-WebRequest -UseBasicParsing "https://raw.githubusercontent.com/microsoft/Windows-Containers/Main/helpful_tools/Install-DockerCE/install-docker-ce.ps1" -o install-docker-ce.ps1
.\install-docker-ce.ps1
# will restart the machine
\end{lstlisting}

\subsection{Usage}
\begin{itemize}
\item connect to the Docker ec2 instance
\item docker pull mcr.microsoft.com/windows/nanoserver:ltsc2022
\item docker pull mcr.microsoft.com/windows/servercore:ltsc2022 (optional)
\item docker run -it <image>
\end{itemize}

\begin{figure}[htbp]
	\includegraphics[width=\textwidth]{static/docker/docker.png}
	\caption{Running windows server container}
\end{figure}
\clearpage

\section{Domain Controller}

\subsection{AWS Configuration}
Instance Name: Domain Controller\\
Instance Type: t3.medium or larger\\
Security Group Ports:
\begin{itemize}
	\item tcp 80
	\item tcp 3389 (RDP)
	\item tcp 0-65535 (All tcp ports)
	\item udp 0-65535 (all udp ports)
	\item icmp -1 (all icmp)
\end{itemize}
Storage: Minimum 50GB SSD\\
Operating System: Windows Server 2019/2022

\subsection{Implementation}

\begin{lstlisting}[language=bash]
Install-WindowsFeature -Name AD-Domain-Services -IncludeManagementTools
Install-ADDSForest -DomainName "example.local" -InstallDNS -SafeModeAdministratorPassword (ConvertTo-SecureString "Mypassword@2025" -AsPlainText -Force) -NoRebootOnCompletion -Force
# Get-Service adws,kdc,netlogon,dns
Restart-Computer
\end{lstlisting}

\subsection{Usage}
\begin{itemize}
\item connect to File Server or other windows server
\item change dns server address to <private ip of domain controller>
\item join the domain 'example.local'
\end{itemize}
\clearpage
\begin{figure}[htbp]
	\includegraphics[width=\textwidth]{static/domain_controller/domain_controller.png}
	\caption{Successfully join the domain}
\end{figure}
\clearpage

\section{DevSpeedLLM}

\subsection{Configuration}
GPU: dual NVIDIA T4 GPUs\\
OS: Linux
\subsection{Implementation}
\begin{lstlisting}[language=python]
from datasets import load_dataset

# train_dataset = load_dataset("trl-lib/Capybara", split="train")
def make_chatml(example):
    return {
        "messages": [ [ {"role": "user", "content": "Summarize the following text:\n\n"  + content },
                             {"role": "assistant", "content":  summary }  ]  +
                             [x for qa in qas for x in (
                             {"role": "user", "content": qa["question"] if qa['question'] != None else ""}, # There are 2 question=None in the dataset
                             {"role": "assistant", "content": qa["answer"]},
                             )] for content, summary, qas in zip(example['content'], example['summary'], example["QAs"])
        ]
    }
dataset = load_dataset("james56352025/devspeedllm-datasets", "rupp_enhanced", split = "train")
train_dataset = dataset.map(make_chatml, batched=True, remove_columns=['summary', 'qa_pairs', 'content', 'QAs'])


from transformers import AutoModelForCausalLM, BitsAndBytesConfig
import torch

# model_id = "Qwen/Qwen2.5-1.5B-Instruct"
# model_id = "mistralai/Mistral-7B-Instruct-v0.1"
# model_id = "google/gemma-3-270m-it"
model_id = "meta-llama/Llama-3.2-3B-Instruct"
output_dir = "lora-out"

four_bit = BitsAndBytesConfig(
        load_in_4bit=True,                        # Load the model in 4-bit precision to save memory
        bnb_4bit_compute_dtype=torch.float16,     # Data type used for internal computations in quantization
        bnb_4bit_use_double_quant=True,           # Use double quantization to improve accuracy
        bnb_4bit_quant_type="nf4"                 # Type of quantization. "nf4" is recommended for recent LLMs
)

model = AutoModelForCausalLM.from_pretrained(
    model_id,
    attn_implementation="sdpa",                   # Change to Flash Attention if GPU has support
    dtype=torch.float16,                          # Change to bfloat16 if GPU has support
    use_cache=True,                               # Whether to cache attention outputs to speed up inference
    # quantization_config=four_bit,
)

from peft import LoraConfig

# You may need to update `target_modules` depending on the architecture of your chosen model.
# For example, different LLMs might have different attention/projection layer names.
peft_config = LoraConfig(
    r=32,
    lora_alpha=32,
    target_modules = ["q_proj", "k_proj", "v_proj", "o_proj", "gate_proj", "up_proj", "down_proj",],
)

import shutil, os, glob
from transformers.trainer_callback import TrainerCallback

class ZipCallback(TrainerCallback):
    # Required empty methods
    def on_init_end(self, args, state, control, **kwargs):
        pass
    def on_train_begin(self, args, state, control, **kwargs):
        pass
    def on_train_end(self, args, state, control, **kwargs):
        pass
    def on_epoch_end(self, args, state, control, **kwargs):
        pass
    # THE ONE WE USE
    def on_save(self, args, state, control, **kwargs):
        # Find latest checkpoint
        # ckpts = sorted(glob.glob(os.path.join(args.output_dir, "checkpoint-*")))
        ckpts = glob.glob(os.path.join(args.output_dir, "checkpoint-*"))
        if not ckpts:
            return control
        # latest = ckpts[-1]
        latest = max(ckpts, key=lambda x: int(x.split("-")[-1]))
        zip_path = os.path.join(os.path.dirname(args.output_dir), os.path.basename(latest) + ".zip")
        if os.path.exists(zip_path):
            print("\n" + f"Already zipped: {zip_path}")
            return control
        shutil.make_archive(os.path.splitext(zip_path)[0], "zip", latest)
        print()
        print(f"Zipped: {zip_path}")
        return control


from trl import SFTConfig, SFTTrainer

training_args = SFTConfig(
    # Training schedule / optimization
    per_device_train_batch_size = 1,      # Batch size per GPU
    gradient_accumulation_steps = 4,      # Gradients are accumulated over multiple steps → effective batch size = 2 * 8 = 16
    warmup_steps = 5,
    num_train_epochs = 5,               # Number of full dataset passes. For shorter training, use `max_steps` instead (this case)
    # max_steps = 2,
    learning_rate = 2e-4,                 # Learning rate for the optimizer
    optim = "paged_adamw_8bit",           # Optimizer
    # Logging / reporting
    logging_steps=1,                      # Log training metrics every N steps
    report_to="trackio",                  # Experiment tracking tool
    trackio_space_id=output_dir,          # HF Space where the experiment tracking will be saved
    output_dir=output_dir,                # Where to save model checkpoints and logs
    # max_length=1024,                      # Maximum input sequence length
    use_liger_kernel=True,                # Enable Liger kernel optimizations for faster training
    activation_offloading=True,           # Offload activations to CPU to reduce GPU memory usage
    gradient_checkpointing=True,          # Save memory by re-computing activations during backpropagation
    # Hub integration
    push_to_hub=False,                    # Automatically push the trained model to the Hugging Face Hub
                                          # The model will be saved under your Hub account in the repository named `output_dir`
    gradient_checkpointing_kwargs={"use_reentrant": False}, # To prevent warning message
)

trainer = SFTTrainer(
    model=model,
    args=training_args,
    train_dataset=train_dataset,
    peft_config=peft_config,
    callbacks = [ZipCallback()],
)

trainer_stats = trainer.train()
trainer.save_model(output_dir)
\end{lstlisting}
\subsection{Usage}
\begin{itemize}
\item ask "tell me about RUPP"
\item ask "list academic staffs of computer science department at RUPP with name"
\item ask "teach me flutter and give example"
\end{itemize}

\begin{figure}[htbp]
	\includegraphics[width=\textwidth]{static/devspeedllm/devspeedllm.png}
	\caption{Chatting with DevSpeedLLM}
\end{figure}

\end{document}
